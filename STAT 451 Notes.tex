\documentclass[12pt,letterpaper,oneside]{article}\usepackage{graphicx, color}
%% maxwidth is the original width if it is less than linewidth
%% otherwise use linewidth (to make sure the graphics do not exceed the margin)
\makeatletter
\def\maxwidth{ %
  \ifdim\Gin@nat@width>\linewidth
    \linewidth
  \else
    \Gin@nat@width
  \fi
}
\makeatother

\IfFileExists{upquote.sty}{\usepackage{upquote}}{}
\definecolor{fgcolor}{rgb}{0.2, 0.2, 0.2}
\newcommand{\hlnumber}[1]{\textcolor[rgb]{0,0,0}{#1}}%
\newcommand{\hlfunctioncall}[1]{\textcolor[rgb]{0.501960784313725,0,0.329411764705882}{\textbf{#1}}}%
\newcommand{\hlstring}[1]{\textcolor[rgb]{0.6,0.6,1}{#1}}%
\newcommand{\hlkeyword}[1]{\textcolor[rgb]{0,0,0}{\textbf{#1}}}%
\newcommand{\hlargument}[1]{\textcolor[rgb]{0.690196078431373,0.250980392156863,0.0196078431372549}{#1}}%
\newcommand{\hlcomment}[1]{\textcolor[rgb]{0.180392156862745,0.6,0.341176470588235}{#1}}%
\newcommand{\hlroxygencomment}[1]{\textcolor[rgb]{0.43921568627451,0.47843137254902,0.701960784313725}{#1}}%
\newcommand{\hlformalargs}[1]{\textcolor[rgb]{0.690196078431373,0.250980392156863,0.0196078431372549}{#1}}%
\newcommand{\hleqformalargs}[1]{\textcolor[rgb]{0.690196078431373,0.250980392156863,0.0196078431372549}{#1}}%
\newcommand{\hlassignement}[1]{\textcolor[rgb]{0,0,0}{\textbf{#1}}}%
\newcommand{\hlpackage}[1]{\textcolor[rgb]{0.588235294117647,0.709803921568627,0.145098039215686}{#1}}%
\newcommand{\hlslot}[1]{\textit{#1}}%
\newcommand{\hlsymbol}[1]{\textcolor[rgb]{0,0,0}{#1}}%
\newcommand{\hlprompt}[1]{\textcolor[rgb]{0.2,0.2,0.2}{#1}}%

\usepackage{framed}
\makeatletter
\newenvironment{kframe}{%
 \def\at@end@of@kframe{}%
 \ifinner\ifhmode%
  \def\at@end@of@kframe{\end{minipage}}%
  \begin{minipage}{\columnwidth}%
 \fi\fi%
 \def\FrameCommand##1{\hskip\@totalleftmargin \hskip-\fboxsep
 \colorbox{shadecolor}{##1}\hskip-\fboxsep
     % There is no \\@totalrightmargin, so:
     \hskip-\linewidth \hskip-\@totalleftmargin \hskip\columnwidth}%
 \MakeFramed {\advance\hsize-\width
   \@totalleftmargin\z@ \linewidth\hsize
   \@setminipage}}%
 {\par\unskip\endMakeFramed%
 \at@end@of@kframe}
\makeatother

\definecolor{shadecolor}{rgb}{.97, .97, .97}
\definecolor{messagecolor}{rgb}{0, 0, 0}
\definecolor{warningcolor}{rgb}{1, 0, 1}
\definecolor{errorcolor}{rgb}{1, 0, 0}
\newenvironment{knitrout}{}{} % an empty environment to be redefined in TeX

\usepackage{alltt}
\newcommand{\university}{Brigham Young University}
\newcommand{\classCode}{STAT 451}
\newcommand{\className}{Applied Bayesian Statistics}
\newcommand{\documentName}{Notes}
\newcommand{\dateTaken}{January 7, 2013}
\newcommand{\semester}{Winter 2013}
\newcommand{\teacher}{Dr. Gil Fellingham}
\newcommand{\docVersion}{00000}
\newcommand{\authorName}{\href{mailto:adam@acjackman.com}{Adam Jackman}}

\input{./Styles/code.tex}
\input{./Styles/pageStyle.tex}

\input{./Styles/mathCommands.tex}




\begin{document}
\input{./Styles/title.tex}
\tableofcontents
\renewcommand{\dateTaken}{January 8, 2013}
\daysep

\section{Class Introduction} % (fold)
\label{sec:class_introduction}
Office Hours Monday

Keep a copy of the code

Preliminary assignment email now.
% section class_introduction (end)

\section{Bayesian Methodology} % (fold)
\label{sec:bayesian_methodology}
Parameterize state of knowledge

Estimate parameters

\begin{align*}
    f(\underline{\theta}|\underline{x})=\frac{L(\underline{x}|\underline{\theta})\prod(\underline{\theta})}{\iint_{\underline{\theta}} L(\underline{x}|\underline{\theta})\prod(\underline{\theta}) \dd \underline{\theta}} \propto L(\underline{x}| \underline{\theta})\prod(\underline{\theta})
\end{align*}
\subsection{Metropolis Sampler} % (fold)
\label{sub:metropolis_sampler}

\begin{align*}
    \pi\left[\frac{1}{\Gamma(\alpha)\beta^{\alpha}}x^{\alpha-1} \exp\left( -\frac{x}{\beta}\right)\right]+ (\pi-1)\left[\frac{1}{\Gamma(\alpha)\beta^{\alpha}}x^{\alpha-1} \exp\left( -\frac{x}{\beta}\right)\right]\\
    \frac{1}{4}\left[\frac{1}{\Gamma(2)3^{2}}x^{2-1} \exp\left( -\frac{x}{3}\right)\right] +
    \frac{3}{4}\left[\frac{1}{\Gamma(4)3^{4}}x^{4-1} \exp\left( -\frac{x}{3}\right)\right]\\
    k=4\Gamma(4) 3^3
\end{align*}
\begin{align*}
    g(x)=k \cdot f(x)= 18 x \exp\left(-\frac{x}{3}\right)+ x^3\exp\left(-\frac{x}{3}\right)
\end{align*}
\subsubsection{Metropolis Sampler Steps} % (fold)
\label{ssub:metropolis_sampler_steps}
\begin{itemize}
    \item Initial $x=x_0$
    \item for $i=1$ to $n$
    \begin{itemize}
        \item generate candidate $c$
        \item $\displaystyle{\frac{g(c)}{g(x_{i-1}} = r}$
        \item let $x_i=c$ with $\prob=\min(1,r)$
    \end{itemize}
\end{itemize}
% subsubsection metropolis_sampler_steps (end)
\subsubsection{Metropolis Code} % (fold)
\label{ssub:metropolis_code}
\begin{knitrout}\scriptsize
\definecolor{shadecolor}{rgb}{0.969, 0.969, 0.969}\color{fgcolor}\begin{kframe}
\begin{alltt}
\hlcomment{# Define g function}
g <- \hlfunctioncall{function}(x) \{
    18 * x * \hlfunctioncall{exp}(-x/3) + x^3 * \hlfunctioncall{exp}(-x/3)
\}

\hlcomment{# Write Sampler}
out <- NULL
out[1] <- 5
candidate.sigma <- 20
counter <- 0
samples <- 10^4
\hlfunctioncall{for} (i in 2:samples) \{
    out[i] <- out[i - 1]
    candidate <- \hlfunctioncall{rnorm}(1, out[i - 1], candidate.sigma)
    \hlfunctioncall{if} (candidate > 0) \{
        r <- \hlfunctioncall{g}(candidate)/\hlfunctioncall{g}(out[i - 1])
        \hlfunctioncall{if} (r > \hlfunctioncall{runif}(1, 0, 1)) \{
            out[i] <- candidate
            counter <- counter + 1
        \}
    \}
\}
\hlfunctioncall{message}(counter/samples)  \hlcomment{# should be about 24-40}
\end{alltt}


{\ttfamily\noindent\itshape\textcolor{messagecolor}{\#\# 0.3285}}\begin{alltt}
\hlfunctioncall{plot}(out, type = \hlstring{"l"}, main = \hlstring{"Trace Plot"})
\end{alltt}
\end{kframe}

{\centering \includegraphics[width=.4\textwidth]{figure/graphic-metropolisSampler} 

}



\end{knitrout}

Trace Plot looks good

\begin{knitrout}\scriptsize
\definecolor{shadecolor}{rgb}{0.969, 0.969, 0.969}\color{fgcolor}\begin{kframe}
\begin{alltt}
xx <- \hlfunctioncall{seq}(0, 50, length = 1e+05)
k <- 4 * \hlfunctioncall{gamma}(4) * 3^3
\hlcomment{# Plot the exact distribution}
\hlfunctioncall{plot}(xx, \hlfunctioncall{g}(xx)/k, lwd = 2, col = \hlstring{"blue"})
\hlcomment{# curve(.25*dgamma(x, shape=2, scale=3)+.75*dgamma(x, shape=4, scale=3),}
\hlcomment{# add=TRUE, col='blue') Both of these should produce the same line}

\hlfunctioncall{lines}(\hlfunctioncall{density}(out), lwd = 2)

\hlcomment{# Exact Mean}
0.25 * (2 * 3) + 0.75 * (4 * 3)
\end{alltt}
\begin{verbatim}
## [1] 10.5
\end{verbatim}
\begin{alltt}

\hlcomment{# Approximate Mean}
\hlfunctioncall{mean}(out)
\end{alltt}
\begin{verbatim}
## [1] 10.35
\end{verbatim}
\end{kframe}

{\centering \includegraphics[width=.4\textwidth]{figure/graphic-MetropolisSamplerPosteriorPlot} 

}



\end{knitrout}

% subsubsection metropolis_code (end)
% subsection metropolis_sampler (end)
% section bayesian_methodology (end)
\renewcommand{\dateTaken}{January 10, 2013}
\daysep

Paul Sabin --- Stat 451 TA\\
Tuesday 2:30-4:00\\
Friday: 9:30-11:00\\
Office: 233\\
Email:\href{mailto:r.paul.sabin@gmail.com}{r.paul.sabin@gmail.com}

\begin{knitrout}\scriptsize
\definecolor{shadecolor}{rgb}{0.969, 0.969, 0.969}\color{fgcolor}\begin{kframe}
\begin{alltt}
\hlcomment{# For Winbugs}
\hlfunctioncall{install.packages}(\hlstring{"ARM"}, \hlstring{"R2WinBUGS"})

\hlcomment{# For jags}
\hlfunctioncall{install.packages}(\hlstring{"R2jags"})
\end{alltt}
\end{kframe}
\end{knitrout}


\section{Two Sample $t$ test} % (fold)
\label{sec:two_sample_t_test}

\begin{align*}
y \sim \mathcal{N}(\mu, \sigma^2)\quad i=1,2\\
f(\mu_1)=\mathcal{N}(\mu=0, \sigma^2=1000)
f(\sigma^2)=\text{Unif}(0, 5000)
\end{align*}
% section two_sample_t_test (end)

PROC MCMC two group $t$ test
\begin{lstlisting}[language=SAS]
data examp2;
infile dataset;
input tmt response;
run;

/*
nmc: Number of Marcov Chains
nbi: Number Burn in

dic: information criteria
*/
proc mcmc data=examp2 outpost=post2 seed=1234 nmc=100000 nbi=5000
                      statistics(alpha=.05)=(summary interval) thin=10
                      moniter=( parms mudif varratio) propcov=quannew
                      diagnostics=(all) dic ;
array mu[2] mu1-mu2;
arra sig2[2] sig21-sig22;
prams: mu: sig2:;
prior mu: ~ normal(0, sd=10000);
prior sig21 ~ unif(9, 5000); /*  gamma(shape=30, scale=50) */
prior sig22 ~ unif(9, 5000);
mudif = mu1 - mu2;
varratio = sig21/sig22;
mm = mu[tmt];
vv = sig2[tmt];
model response ~ normal(mm, var=vv);
run;

proc export data=post2 outfile='twogroups.csv' dbms=csv replace;
run;
\end{lstlisting}

\begin{description}
    \item[HPD --- Highest Posterior Density] the smallest interval
\end{description}

Fischer variance problem --- a two sample $t$ test with unequal variances


% <<engine='SAS'>>=
% data foo;
% input a b c;
% datalines;
% 1 2 3
% 4 5 6
% 7 8 9
% ;
% run;
% @

Check the posterior autocorolations

\renewcommand{\dateTaken}{January 15, 2013}
\daysep
\begin{knitrout}\scriptsize
\definecolor{shadecolor}{rgb}{0.969, 0.969, 0.969}\color{fgcolor}\begin{kframe}
\begin{alltt}
\hlfunctioncall{library}(rjags)
\hlfunctioncall{library}(R2jags)
\end{alltt}
\end{kframe}
\end{knitrout}

JAGS uses precisions, not variances. Precisions are $\frac{1}{\sigma^2}$

\lstinputlisting[language=R]{code/TwoSampleJAGS.R}


\renewcommand{\dateTaken}{January 17, 2013}
\daysep

You need to think about priors

\section{Diagnostics} % (fold)
\label{sec:diagnostics}
\begin{knitrout}\scriptsize
\definecolor{shadecolor}{rgb}{0.969, 0.969, 0.969}\color{fgcolor}\begin{kframe}
\begin{alltt}
\hlfunctioncall{library}(R2jags)
mdl <- "
    model \{
        \hlfunctioncall{for} (i in 1:33) \{
            y[i] ~ \hlfunctioncall{dnorm}(mu[tmt[i]], prec[tmt[i]])
        \}

        \hlfunctioncall{for} (i in 1:2) \{
            mu[i] ~ \hlfunctioncall{dnorm}(0,0.000001)
            vr[i] ~ \hlfunctioncall{dunif}(0,5000)
            prec[i] <- 1/vr[i]
        \}
    \}
"
groups <- \hlfunctioncall{read.table}(\hlstring{"data/01twogroups.dat"}, col.names=\hlfunctioncall{c}(\hlstring{"tmt"}, \hlstring{"y"}))
\hlfunctioncall{writeLines}(mdl, \hlstring{"code/twogroups.jags"})
tmt <- groups$tmt
y <- groups$y
\hlcomment{# Data to go into jags}
data.jags <- \hlfunctioncall{c}(\hlstring{'tmt'},\hlstring{'y'})
\hlcomment{# what parameters to keep track of}
parms <- \hlfunctioncall{c}(\hlstring{'mu'},\hlstring{'vr'})
\hlcomment{# Initial Values}
innts <- \hlfunctioncall{function}() \{\hlfunctioncall{list}(\hlstring{'mu'} = \hlfunctioncall{rnorm}(2,125,5), \hlstring{'vr'} = \hlfunctioncall{runif}(2,0,5000))\}
twogroups.sim <- \hlfunctioncall{jags}(data=data.jags, parameters.to.save=parms, inits=innts, model.file=\hlstring{"code/twogroups.jags"},
    n.iter=6000, n.burnin=1000, n.chains=1, n.thin=1)
\end{alltt}


{\ttfamily\noindent\itshape\textcolor{messagecolor}{\#\# module glm loaded}}\begin{verbatim}
## Compiling model graph
##    Resolving undeclared variables
##    Allocating nodes
##    Graph Size: 76
## 
## Initializing model
\end{verbatim}
\begin{alltt}

twogroups.sim
\end{alltt}
\begin{verbatim}
## Inference for Bugs model at "code/twogroups.jags", fit using jags,
##  1 chains, each with 6000 iterations (first 1000 discarded)
##  n.sims = 5000 iterations saved
##          mu.vect sd.vect   2.5%    25%    50%    75%  97.5%
## mu[1]     134.75   6.499 122.11 130.61 134.70 138.87  147.8
## mu[2]     121.40   2.110 117.16 120.09 121.38 122.73  125.6
## vr[1]     797.45 333.228 381.36 573.92 723.20 930.48 1659.5
## vr[2]      63.12  31.741  26.22  41.34  55.22  76.11  146.2
## deviance  271.55   3.652 266.96 268.84 270.74 273.37  280.5
## 
## DIC info (using the rule, pD = var(deviance)/2)
## pD = 6.7 and DIC = 278.2
## DIC is an estimate of expected predictive error (lower deviance is better).
\end{verbatim}
\begin{alltt}

sims <- \hlfunctioncall{as.mcmc}(twogroups.sim)
\hlfunctioncall{plot}(sims, auto.layout=FALSE, ask=FALSE)

\hlcomment{# plot(sims[,2], type='l')}
\end{alltt}
\end{kframe}

{\centering \includegraphics[width=.4\textwidth]{figure/graphic-unnamed-chunk-31} 
\includegraphics[width=.4\textwidth]{figure/graphic-unnamed-chunk-32} 
\includegraphics[width=.4\textwidth]{figure/graphic-unnamed-chunk-33} 
\includegraphics[width=.4\textwidth]{figure/graphic-unnamed-chunk-34} 
\includegraphics[width=.4\textwidth]{figure/graphic-unnamed-chunk-35} 
\includegraphics[width=.4\textwidth]{figure/graphic-unnamed-chunk-36} 
\includegraphics[width=.4\textwidth]{figure/graphic-unnamed-chunk-37} 
\includegraphics[width=.4\textwidth]{figure/graphic-unnamed-chunk-38} 
\includegraphics[width=.4\textwidth]{figure/graphic-unnamed-chunk-39} 
\includegraphics[width=.4\textwidth]{figure/graphic-unnamed-chunk-310} 

}



\end{knitrout}


First diagnostic: Look at your trace plot

Second diagnostic: look at auto corilation
\begin{knitrout}\scriptsize
\definecolor{shadecolor}{rgb}{0.969, 0.969, 0.969}\color{fgcolor}\begin{kframe}
\begin{alltt}
\hlfunctioncall{autocorr}(sims)
\end{alltt}
\begin{verbatim}
## , , deviance
## 
##         deviance      mu[1]     mu[2]      vr[1]      vr[2]
## Lag 0   1.000000  0.0397216 -0.025591  0.5294713  0.5567647
## Lag 1   0.009287  0.0178228 -0.001470 -0.0079258  0.0045190
## Lag 5   0.008346  0.0222125 -0.004006  0.0258781  0.0005049
## Lag 10 -0.003025  0.0082836 -0.023008  0.0244412 -0.0188622
## Lag 50 -0.004687 -0.0007478 -0.009660 -0.0006706  0.0165176
## 
## , , mu[1]
## 
##        deviance      mu[1]     mu[2]      vr[1]     vr[2]
## Lag 0  0.039722  1.0000000 -0.019048  0.0242092  0.013122
## Lag 1  0.010297 -0.0176032 -0.018843 -0.0115282 -0.001713
## Lag 5  0.009260  0.0203083 -0.010197  0.0290402 -0.001748
## Lag 10 0.002206  0.0183513  0.048567 -0.0005999  0.015041
## Lag 50 0.001726 -0.0001966  0.002256  0.0190355 -0.014390
## 
## , , mu[2]
## 
##         deviance      mu[1]     mu[2]     vr[1]     vr[2]
## Lag 0  -0.025591 -0.0190482  1.000000 -0.002025 -0.017025
## Lag 1  -0.007725 -0.0083044  0.016805 -0.021768  0.005684
## Lag 5   0.030497 -0.0280916 -0.008886  0.010412  0.008702
## Lag 10  0.015048 -0.0139730  0.026550  0.019316 -0.003001
## Lag 50  0.010831  0.0004535 -0.001931 -0.006375  0.012813
## 
## , , vr[1]
## 
##         deviance     mu[1]     mu[2]     vr[1]     vr[2]
## Lag 0   0.529471  0.024209 -0.002025  1.000000  0.006924
## Lag 1   0.012432  0.008644  0.003187  0.013754  0.002647
## Lag 5  -0.006325 -0.007802 -0.013119 -0.001014  0.001037
## Lag 10 -0.001276 -0.022398 -0.016429  0.018451 -0.021251
## Lag 50  0.020901 -0.010891 -0.008754  0.011124  0.031264
## 
## , , vr[2]
## 
##         deviance    mu[1]    mu[2]    vr[1]      vr[2]
## Lag 0   0.556765 0.013122 -0.01702 0.006924  1.0000000
## Lag 1  -0.002083 0.002571 -0.00108 0.004606  0.0085805
## Lag 5   0.007769 0.018526 -0.01204 0.033278 -0.0131159
## Lag 10  0.020122 0.035722 -0.01813 0.019459 -0.0003303
## Lag 50 -0.019064 0.023849 -0.01956 0.001463 -0.0069104
\end{verbatim}
\begin{alltt}
\hlfunctioncall{autocorr.plot}(sims)
\end{alltt}
\end{kframe}

{\centering \includegraphics[width=.4\textwidth]{figure/graphic-unnamed-chunk-4} 

}



\end{knitrout}


A good autocorrelation plot shows the first bar being tall, but the rest being relatively small.

Fourth diagnostic: Effective Sample size
\begin{knitrout}\scriptsize
\definecolor{shadecolor}{rgb}{0.969, 0.969, 0.969}\color{fgcolor}\begin{kframe}
\begin{alltt}
\hlfunctioncall{effectiveSize}(sims)
\end{alltt}
\begin{verbatim}
## deviance    mu[1]    mu[2]    vr[1]    vr[2] 
##     5000     5000     5000     5000     4840
\end{verbatim}
\end{kframe}
\end{knitrout}

these should be relatively close to the actual sample size

fifth diagnostic: Raftery Lewis Diagnostic
\begin{knitrout}\scriptsize
\definecolor{shadecolor}{rgb}{0.969, 0.969, 0.969}\color{fgcolor}\begin{kframe}
\begin{alltt}
\hlfunctioncall{raftery.diag}(sims)
\end{alltt}
\begin{verbatim}
## 
## Quantile (q) = 0.025
## Accuracy (r) = +/- 0.005
## Probability (s) = 0.95 
##                                                 
##           Burn-in  Total Lower bound  Dependence
##           (M)      (N)   (Nmin)       factor (I)
##  deviance 2        3741  3746         0.999     
##  mu[1]    2        3741  3746         0.999     
##  mu[2]    2        3930  3746         1.050     
##  vr[1]    2        3803  3746         1.020     
##  vr[2]    2        3561  3746         0.951
\end{verbatim}
\end{kframe}
\end{knitrout}

Check if you are within .005 (in the tails) with high probability

Dependence Factor should be less than 5

These five are the most commonly used, but there are others
\begin{knitrout}\scriptsize
\definecolor{shadecolor}{rgb}{0.969, 0.969, 0.969}\color{fgcolor}\begin{kframe}
\begin{alltt}
\hlfunctioncall{geweke.diag}(sims)
\end{alltt}
\begin{verbatim}
## 
## Fraction in 1st window = 0.1
## Fraction in 2nd window = 0.5 
## 
## deviance    mu[1]    mu[2]    vr[1]    vr[2] 
##  -0.6146  -1.7786  -0.8627   0.7567   0.1023
\end{verbatim}
\end{kframe}
\end{knitrout}

Tests for drift in the samples. does something like a $t$ test on the first 10\% and the last 50\% of the chain.

Heild
\begin{knitrout}\scriptsize
\definecolor{shadecolor}{rgb}{0.969, 0.969, 0.969}\color{fgcolor}\begin{kframe}
\begin{alltt}
\hlfunctioncall{heidel.diag}(sims)
\end{alltt}
\begin{verbatim}
##                                        
##          Stationarity start     p-value
##          test         iteration        
## deviance passed       1         0.822  
## mu[1]    passed       1         0.140  
## mu[2]    passed       1         0.398  
## vr[1]    passed       1         0.684  
## vr[2]    passed       1         0.636  
##                                   
##          Halfwidth Mean  Halfwidth
##          test                     
## deviance passed    271.5 0.1012   
## mu[1]    passed    134.7 0.1802   
## mu[2]    passed    121.4 0.0585   
## vr[1]    passed    797.5 9.2366   
## vr[2]    passed     63.1 0.8942
\end{verbatim}
\end{kframe}
\end{knitrout}

test of stationarity, or convergence

both the stationarity and half width mean test should pass.


Gelmen needs chains started at different points
% section diagnostics (end)
\renewcommand{\dateTaken}{January 22, 2013}
\daysep
\section{Posterior Predictives} % (fold)
\label{sec:posterior_sredictives}
Liklihood $g_{i1}\sim \mathcal{N}(\mu_1, \sigma^2_1$, $g_{i2}\sim \mathcal{N}(\mu_2, \sigma^2_2$

Get a new y by plugging the sampled values of the parameters into a random sampler for the likelihood
\begin{lstlisting}
ynew_1 <- rnorm(1,mu1[1],sigma1[1])
\end{lstlisting}

\begin{tabular}{ccccc}
 $\mu_1$ & $\sigma^2_1$ & $\mu_2$ & $\sigma^2_2$ & $y_{\text{new}1}$\\
 \vdots & \vdots & \vdots & \vdots & \vdots
\end{tabular}

\begin{knitrout}\scriptsize
\definecolor{shadecolor}{rgb}{0.969, 0.969, 0.969}\color{fgcolor}\begin{kframe}
\begin{alltt}
ratios <- \hlfunctioncall{read.table}(\hlstring{"../Homework/01/monkeyratio.dat"})[,1]
\hlfunctioncall{library}(R2jags)
N <- \hlfunctioncall{length}(ratios)

mdl <- "
    model \{
        \hlfunctioncall{for} (i in 1:N) \{
            ratios[i] ~ \hlfunctioncall{dbeta}(alpha, beta)
        \}

        alpha ~ \hlfunctioncall{dgamma}(5, .025)
        beta ~ \hlfunctioncall{dgamma}(5, .025)
    \}
"
\hlfunctioncall{writeLines}(mdl, \hlstring{"code/hw1.jags"})

jags.data <- \hlfunctioncall{c}(\hlstring{"ratios"}, \hlstring{"N"})
jags.params <- \hlfunctioncall{c}(\hlstring{"alpha"}, \hlstring{"beta"})
jags.inits <- \hlfunctioncall{function}() \{
    \hlfunctioncall{list}(\hlstring{'alpha'} <- \hlfunctioncall{rgamma}(5,.025), \hlstring{'beta'} <- \hlfunctioncall{rgamma}(5,.025))
\}
\hlfunctioncall{set.seed}(5487)

ratios.sim <- \hlfunctioncall{jags}(data=jags.data,
                   parameters.to.save=jags.params,
\hlcomment{                   # inits=jags.inits,}
                   model.file=\hlstring{"code/hw1.jags"},
                   n.iter=6000, n.burnin=1000, n.chains=1, n.thin=1)
\end{alltt}


{\ttfamily\noindent\itshape\textcolor{messagecolor}{\#\# module glm loaded}}\begin{verbatim}
## Compiling model graph
##    Resolving undeclared variables
##    Allocating nodes
##    Graph Size: 10
## 
## Initializing model
\end{verbatim}
\begin{alltt}

sims <- \hlfunctioncall{as.mcmc}(ratios.sim)

\hlcomment{# autocorr.plot(sims)}
alpha <- sims[,1]
beta <- sims[,2]
ynew <- \hlfunctioncall{rbeta}(10000,alpha,beta)

\hlcomment{# Plot of the density of the mean and the posterior predictive}
\hlfunctioncall{plot}(\hlfunctioncall{density}(alpha/(alpha+beta)), xlim=\hlfunctioncall{c}(.6,.9))
\hlfunctioncall{lines}(\hlfunctioncall{density}(ynew), col=\hlstring{'blue'}) \hlcomment{# Plot of the posterior predictive}
\end{alltt}
\end{kframe}

{\centering \includegraphics[width=.4\textwidth]{figure/graphic-unnamed-chunk-9} 

}



\end{knitrout}


 We don't make up data we get it from ``Pedagogically flexible data acquisition.''---Dr Tolly.


You can use other priors, they just may not be as easy for example:


% section posterior_predictives (end)
\section{Linear Regression} % (fold)
\label{sec:linear_regression}
Likelihood \[ y_{i} ~ \mathcal{N}(\mu_{i}, \sigma^2)\]

The mean is different for every data point, but the variance is constant.

\begin{align*}
    \mu_1 \leftarrow \beta_{0} + \beta_{1} x\\
    \beta_{0} \sim \mathcal{N}(,\tau=.001)\\
    \beta_{1} \sim \mathcal{N}(\mu=0,\tau=.001)\\
    \sigma^{2} \sim \\
\end{align*}
\renewcommand{\dateTaken}{January 24, 2013}
\daysep

\begin{knitrout}\scriptsize
\definecolor{shadecolor}{rgb}{0.969, 0.969, 0.969}\color{fgcolor}\begin{kframe}
\begin{alltt}
jags.modelfile <- \hlstring{"code/linreg.jags"}
mdl <- \hlstring{"\textbackslash{}nmodel \{\textbackslash{}\hlfunctioncall{nfor} (i in 1:N) \{\textbackslash{}ny[i] ~ \hlfunctioncall{dnorm}(mu[i], prec)\textbackslash{}nmu[i] <- beta_0 + beta_1 * x[i]\textbackslash{}n\}\textbackslash{}\hlfunctioncall{ninvisible}(\textbackslash{}"}.BeGiN_TiDy_IdEnTiFiEr_HaHaHa.HaHaHa_EnD_TiDy_IdEnTiFiEr\textbackslash{}\hlstring{")\textbackslash{}nbeta_0 ~ \hlfunctioncall{dnorm}(0, .00001)\textbackslash{}nbeta_1 ~ \hlfunctioncall{dnorm}(0, .001)\textbackslash{}nsigma2 ~ \hlfunctioncall{dgamma}(1.1, .5)\textbackslash{}nprec <- 1/sigma2\textbackslash{}n\}\textbackslash{}n"}
\hlfunctioncall{writeLines}(mdl, \hlstring{"code/linreg.jags"})


linregdata <- \hlfunctioncall{read.table}(\hlstring{"data/02linreg.dat"}, col.names = \hlfunctioncall{c}(\hlstring{"lotsize"}, \hlstring{"manhours"}))
N <- \hlfunctioncall{nrow}(linregdata)
y <- linregdata[, 2]
x <- linregdata[, 1]
jags.data <- \hlfunctioncall{c}(\hlstring{"y"}, \hlstring{"x"}, \hlstring{"N"})
jags.params <- \hlfunctioncall{c}(\hlstring{"beta_0"}, \hlstring{"beta_1"}, \hlstring{"sigma2"})


\hlfunctioncall{set.seed}(5487)
linreg.sim <- \hlfunctioncall{jags}(data = jags.data, parameters.to.save = jags.params, model.file = \hlstring{"code/linreg.jags"}, 
    n.iter = 10000, n.burnin = 5000, n.chains = 1, n.thin = 1)
\end{alltt}


{\ttfamily\noindent\itshape\textcolor{messagecolor}{\#\# module glm loaded}}

{\ttfamily\noindent\bfseries\textcolor{errorcolor}{\#\# Error: \\\#\# Error parsing model file:\\\#\# syntax error on line 7 near """}}\begin{alltt}
linreg.sim
\end{alltt}


{\ttfamily\noindent\bfseries\textcolor{errorcolor}{\#\# Error: object 'linreg.sim' not found}}\end{kframe}
\end{knitrout}


Let's check for convergence
\begin{knitrout}\scriptsize
\definecolor{shadecolor}{rgb}{0.969, 0.969, 0.969}\color{fgcolor}\begin{kframe}
\begin{alltt}
sims <- \hlfunctioncall{as.mcmc}(linreg.sim)
\end{alltt}


{\ttfamily\noindent\bfseries\textcolor{errorcolor}{\#\# Error: object 'linreg.sim' not found}}\begin{alltt}
\hlcomment{# Trace Plots}
\hlfunctioncall{plot}(sims, auto.layout = FALSE, ask = FALSE)
\hlcomment{# Autocorrelation plots}
\hlfunctioncall{autocorr.plot}(sims, ask = FALSE)
\hlfunctioncall{effectiveSize}(sims)
\end{alltt}
\begin{verbatim}
##    alpha     beta deviance 
##     5000     5000     5000
\end{verbatim}
\begin{alltt}
\hlfunctioncall{raftery.diag}(sims)
\end{alltt}
\begin{verbatim}
## 
## Quantile (q) = 0.025
## Accuracy (r) = +/- 0.005
## Probability (s) = 0.95 
##                                                 
##           Burn-in  Total Lower bound  Dependence
##           (M)      (N)   (Nmin)       factor (I)
##  alpha    2        3741  3746         0.999     
##  beta     2        3680  3746         0.982     
##  deviance 2        3680  3746         0.982
\end{verbatim}
\end{kframe}

{\centering \includegraphics[width=.4\textwidth]{figure/graphic-unnamed-chunk-121} 
\includegraphics[width=.4\textwidth]{figure/graphic-unnamed-chunk-122} 
\includegraphics[width=.4\textwidth]{figure/graphic-unnamed-chunk-123} 
\includegraphics[width=.4\textwidth]{figure/graphic-unnamed-chunk-124} 
\includegraphics[width=.4\textwidth]{figure/graphic-unnamed-chunk-125} 
\includegraphics[width=.4\textwidth]{figure/graphic-unnamed-chunk-126} 
\includegraphics[width=.4\textwidth]{figure/graphic-unnamed-chunk-127} 

}



\end{knitrout}


Here's the frequentest method for an ANOVA
\begin{knitrout}\scriptsize
\definecolor{shadecolor}{rgb}{0.969, 0.969, 0.969}\color{fgcolor}\begin{kframe}
\begin{alltt}
fit.1 <- \hlfunctioncall{lm}(y ~ x)
\hlfunctioncall{anova}(fit.1)
\end{alltt}
\begin{verbatim}
## Analysis of Variance Table
## 
## Response: y
##           Df Sum Sq Mean Sq F value Pr(>F)    
## x          1  13600   13600    1813  1e-10 ***
## Residuals  8     60       7                   
## ---
## Signif. codes:  0 '***' 0.001 '**' 0.01 '*' 0.05 '.' 0.1 ' ' 1
\end{verbatim}
\end{kframe}
\end{knitrout}


to check the value of the slope
\begin{knitrout}\scriptsize
\definecolor{shadecolor}{rgb}{0.969, 0.969, 0.969}\color{fgcolor}\begin{kframe}
\begin{alltt}
beta0 <- sims[, 1]
beta1 <- sims[, 2]
sigma2 <- sims[, 4]
\end{alltt}


{\ttfamily\noindent\bfseries\textcolor{errorcolor}{\#\# Error: subscript out of bounds}}\begin{alltt}

\hlcomment{# Probability slope is greater than 0}
\hlfunctioncall{mean}(beta1 > 0)
\end{alltt}
\begin{verbatim}
## [1] 1
\end{verbatim}
\begin{alltt}

\hlcomment{# Plot the density of the slope of the line}
\hlfunctioncall{plot}(\hlfunctioncall{density}(beta1))
\end{alltt}
\end{kframe}

{\centering \includegraphics[width=.4\textwidth]{figure/graphic-unnamed-chunk-14} 

}



\end{knitrout}


Plot of regression
\begin{knitrout}\scriptsize
\definecolor{shadecolor}{rgb}{0.969, 0.969, 0.969}\color{fgcolor}\begin{kframe}
\begin{alltt}
\hlfunctioncall{plot}(x, y, pch = 20)
\hlfunctioncall{abline}(\hlfunctioncall{mean}(beta0), \hlfunctioncall{mean}(beta1))
xx <- \hlfunctioncall{seq}(\hlfunctioncall{min}(x), \hlfunctioncall{max}(x), by = 0.01)

ans <- beta0 + \hlfunctioncall{outer}(beta1, xx, \hlstring{"*"})
\hlfunctioncall{lines}(xx, \hlfunctioncall{apply}(ans, 2, quantile, 0.025), col = \hlstring{"blue"})
\hlfunctioncall{lines}(xx, \hlfunctioncall{apply}(ans, 2, quantile, 0.975), col = \hlstring{"blue"})

\hlcomment{# generate a new data set with the first set of}
\hlfunctioncall{points}(xx, beta0[1] + beta1[1] * xx + \hlfunctioncall{rnorm}(\hlfunctioncall{length}(xx), 0, \hlfunctioncall{sqrt}(sigma2[1])), pch = 20)
\end{alltt}


{\ttfamily\noindent\bfseries\textcolor{errorcolor}{\#\# Error: object 'sigma2' not found}}\end{kframe}

{\centering \includegraphics[width=.4\textwidth]{figure/graphic-unnamed-chunk-15} 

}



\end{knitrout}


Homework Plot the prediction interval from the problem in class

generate a normal using  ans each ans and sigma
% section linear_regression (end)
\renewcommand{\dateTaken}{January 29, 2013}
\daysep
\section{Multiple Linear Regression} % (fold)
\label{sec:multiple_linear_regression}
\lstinputlisting[language=SAS]{code/MultipleRegression.sas}
Raftery-lewis diagnostics should be less than 5.

Let the variance increase proportionaly with the $x$s

\begin{knitrout}\scriptsize
\definecolor{shadecolor}{rgb}{0.969, 0.969, 0.969}\color{fgcolor}\begin{kframe}
\begin{alltt}
vo2 <- \hlfunctioncall{read.table}(\hlstring{"data/03vo2.dat"}, skip = 1, col.names = \hlfunctioncall{c}(\hlstring{"ID"}, \hlstring{"Gender"}, \hlstring{"Age1"}, 
    \hlstring{"BMI"}, \hlstring{"MPH"}, \hlstring{"HR"}, \hlstring{"RPE"}, \hlstring{"MaxVO2ML"}))
\hlfunctioncall{plot}(vo2$MPH, vo2$MaxVO2ML)
\hlfunctioncall{plot}(vo2$BMI, vo2$MaxVO2ML)
\hlfunctioncall{plot}(vo2$Gender, vo2$MaxVO2ML)
\end{alltt}
\end{kframe}

{\centering \includegraphics[width=.4\textwidth]{figure/graphic-unnamed-chunk-161} 
\includegraphics[width=.4\textwidth]{figure/graphic-unnamed-chunk-162} 
\includegraphics[width=.4\textwidth]{figure/graphic-unnamed-chunk-163} 

}



\end{knitrout}



DIC Deviance information criteria (calculated in formula~\ref{eq:deviance})
\begin{align} \label{eq:deviance}
    \text{deviance} = -2\log(\mathcal{L})
\end{align}

\renewcommand{\dateTaken}{January 31, 2013}
\daysep
\begin{knitrout}\scriptsize
\definecolor{shadecolor}{rgb}{0.969, 0.969, 0.969}\color{fgcolor}\begin{kframe}
\begin{alltt}
\hlcomment{# Using the VO2 Data}
N <- \hlfunctioncall{nrow}(vo2)
mdl <- '
model \{
    \hlfunctioncall{for}( i in 1:N)\{
        y[i] ~ \hlfunctioncall{dnorm}(mu[i], prec)
        mu[i] <- b_0 +
                 b_gen*gender[i] +
                 b_bmi*bmi[i] +
                 b_mph*mph[i] +
                 b_hr*hr[i] +
                 b_rpe*rpe[i] +
\hlcomment{                 # b_gen_bmi*gen_bmi[i] +}
\hlcomment{                 # b_gen_mph*gen_mph[i] +}
\hlcomment{                 # b_gen_hr*gen_hr[i] +}
\hlcomment{                 # b_gen_rpe*gen_rpe[i] +}
\hlcomment{                 # b_bmi_mph*bmi_mph[i] +}
\hlcomment{                 # b_bmi_hr*bmi_hr[i] +}
\hlcomment{                 # b_bmi_rpe*bmi_rpe[i] +}
                 b_mph_hr*mph_hr[i] \hlcomment{# +}
\hlcomment{                 # b_mph_rpe*mph_rpe[i] +}
\hlcomment{                 # b_hr_rpe*hr_rpe[i]}
    \}

    b_0   ~ \hlfunctioncall{dnorm}(0, .00001);
    b_gen ~ \hlfunctioncall{dnorm}(0, .001);
    b_bmi ~ \hlfunctioncall{dnorm}(0, .001);
    b_mph ~ \hlfunctioncall{dnorm}(0, .001);
    b_hr  ~ \hlfunctioncall{dnorm}(0, .001);
    b_rpe ~ \hlfunctioncall{dnorm}(0, .001);

\hlcomment{    # b_gen_bmi ~ dnorm(0, .001);}
\hlcomment{    # b_gen_mph ~ dnorm(0, .001);}
\hlcomment{    # b_gen_hr  ~ dnorm(0, .001);}
\hlcomment{    # b_gen_rpe ~ dnorm(0, .001);}
\hlcomment{    # b_bmi_mph ~ dnorm(0, .001);}
\hlcomment{    # b_bmi_hr  ~ dnorm(0, .001);}
\hlcomment{    # b_bmi_rpe ~ dnorm(0, .001);}
    b_mph_hr  ~ \hlfunctioncall{dnorm}(0, .001);
\hlcomment{    # b_mph_rpe ~ dnorm(0, .001);}
\hlcomment{    # b_hr_rpe  ~ dnorm(0, .001);}

    vr ~ \hlfunctioncall{dgamma}(2, .05)
    prec <- 1/vr
\}
'
\hlfunctioncall{writeLines}(mdl, \hlstring{"code/multipleRegression.jags"})

y <- vo2[,8]
gender <- vo2[,2]
bmi <- vo2[,4]
hr <- vo2[,6]
mph <- vo2[,5]
rpe <- vo2[,7]

\hlcomment{# gen_bmi <- gender*bmi}
\hlcomment{# gen_hr <- gender*hr}
\hlcomment{# gen_mph <- gender*mph}
\hlcomment{# gen_rpe <- gender*rpe}
\hlcomment{# bmi_mph <- bmi*mph}
\hlcomment{# bmi_hr <- bmi*hr}
\hlcomment{# bmi_rpe <- bmi*rpe}
mph_hr <- mph*hr
\hlcomment{# mph_rpe <- mph*rpe}
\hlcomment{# hr_rpe <- hr*rpe}

jags.data <- \hlfunctioncall{c}(
\hlstring{"N"},
\hlstring{"y"},
\hlstring{"gender"},
\hlstring{"bmi"},
\hlstring{"hr"},
\hlstring{"mph"},
\hlstring{"rpe"},
\hlcomment{# "gen_bmi",}
\hlcomment{# "gen_hr",}
\hlcomment{# "gen_mph",}
\hlcomment{# "gen_rpe",}
\hlcomment{# "bmi_mph",}
\hlcomment{# "bmi_hr" #,}
\hlcomment{# "bmi_rpe",}
\hlstring{"mph_hr"} #,
\hlcomment{# "mph_rpe",}
\hlcomment{# "hr_rpe",}
)
jags.params <- \hlfunctioncall{c}(
\hlstring{"b_0"},
\hlstring{"b_gen"},
\hlstring{"b_bmi"},
\hlstring{"b_hr"},
\hlstring{"b_mph"},
\hlstring{"b_rpe"},
\hlcomment{# "b_gen_bmi",}
\hlcomment{# "b_gen_hr",}
\hlcomment{# "b_gen_mph",}
\hlcomment{# "b_gen_rpe",}
\hlcomment{# "b_bmi_mph",}
\hlcomment{# "b_bmi_hr",}
\hlcomment{# "b_bmi_rpe",}
\hlstring{"b_mph_hr"},
\hlcomment{# "b_mph_rpe",}
\hlstring{"vr"}
)

\hlfunctioncall{set.seed}(1234)
multr.sim <- \hlfunctioncall{jags}(data=jags.data,
                   parameters.to.save=jags.params,
                   model.file=\hlstring{"code/multipleRegression.jags"},
                   n.iter=10000, n.burnin=5000, n.chains=1, n.thin=1)
\end{alltt}


{\ttfamily\noindent\itshape\textcolor{messagecolor}{\#\# module glm loaded}}\begin{verbatim}
## Compiling model graph
##    Resolving undeclared variables
##    Allocating nodes
##    Graph Size: 1294
## 
## Initializing model
\end{verbatim}
\end{kframe}
\end{knitrout}


\begin{tabular}{cc}
Full & 756.4\\
-hr & 764.4\\
-rpe & 759.2\\
+ all two way interactions & 766.7\\
\end{tabular}

\begin{align*}
    IC= -2\log(\mathcal{L}) + \text{penalty}(\text{parameters})\\
    \text{Deviance} = -2 \log(\mathcal{L})\\
    DIC = -2\log(\mathcal{L}) + pD\\
    pD = \frac{\text{VAR}(\text{deviance})}{2}
\end{align*}
% section multiple_linear_regression (end)
\section{ANOVA} % (fold)
\label{sec:anova}
Cell means model of ANOVA


\begin{tabular}{c|c|c|c}
Treatement 1& Treatement 2 & Treatement 3 & Treatement 4\\
\hline
$y_{11}$ & $y_{21}$ & $y_{31}$ & $y_{41}$ \\
$y_{12}$ & $y_{22}$ & $y_{32}$ & $y_{42}$ \\
$y_{13}$ & $y_{23}$ & $y_{33}$ & $y_{43}$ \\
$y_{14}$ & $y_{24}$ & $y_{34}$ & $y_{44}$ \\
\end{tabular}

so given this data we say that
$y_{1}\sim \mathrm{N}(\mu,\sigma^{2})$

\begin{knitrout}\scriptsize
\definecolor{shadecolor}{rgb}{0.969, 0.969, 0.969}\color{fgcolor}\begin{kframe}
\begin{alltt}
anova <- \hlfunctioncall{read.table}(\hlstring{"data/04anova.dat"} ,col.names=\hlfunctioncall{c}(\hlstring{"tmt"},\hlstring{"response"},\hlstring{"tmt1"}))
N <- \hlfunctioncall{nrow}(anova)
mdl <- 'model \{
\hlcomment{    # Likelihood}
    \hlfunctioncall{for} (i in 1:N)\{
        response[i] ~ \hlfunctioncall{dnorm}(mu[tmt[i]], prec)
    \}

\hlcomment{    # Create priors for each treatment}
    \hlfunctioncall{for}(i in 1:4)\{
        mu[i] ~ \hlfunctioncall{dnorm}(15,.0001)
    \}
    prec <- 1/vv
    vv ~ \hlfunctioncall{dgamma}(1.1,.1)
\}
'
\hlfunctioncall{writeLines}(mdl, \hlstring{"code/ANOVAmodel.jags"})
response <- anova$response
tmt <- anova$tmt
\end{alltt}
\end{kframe}
\end{knitrout}

\begin{knitrout}\scriptsize
\definecolor{shadecolor}{rgb}{0.969, 0.969, 0.969}\color{fgcolor}\begin{kframe}
\begin{alltt}
jags.data <- \hlfunctioncall{c}(\hlstring{"N"}, \hlstring{"response"}, \hlstring{"tmt"})
jags.params <- \hlfunctioncall{c}(\hlstring{"mu"}, \hlstring{"vv"})
anova.sim <- \hlfunctioncall{jags}(data = jags.data, parameters.to.save = jags.params, model.file = \hlstring{"code/ANOVAmodel.jags"}, 
    n.iter = 12000, n.chains = 1, n.thin = 1)
\end{alltt}
\begin{verbatim}
## Compiling model graph
##    Resolving undeclared variables
##    Allocating nodes
##    Graph Size: 68
## 
## Initializing model
\end{verbatim}
\begin{alltt}
sims <- \hlfunctioncall{as.mcmc}(anova.sim)
\hlfunctioncall{plot}(sims, auto.layout = FALSE, ask = FALSE)
\hlfunctioncall{autocorr.plot}(sims)
\hlfunctioncall{raftery.diag}(sims)
\end{alltt}
\begin{verbatim}
## 
## Quantile (q) = 0.025
## Accuracy (r) = +/- 0.005
## Probability (s) = 0.95 
##                                                 
##           Burn-in  Total Lower bound  Dependence
##           (M)      (N)   (Nmin)       factor (I)
##  deviance 2        3761  3746         1.00      
##  mu[1]    2        3710  3746         0.99      
##  mu[2]    2        3761  3746         1.00      
##  mu[3]    2        3761  3746         1.00      
##  mu[4]    2        3813  3746         1.02      
##  vv       3        4028  3746         1.08
\end{verbatim}
\begin{alltt}
\hlfunctioncall{effectiveSize}(sims)
\end{alltt}
\begin{verbatim}
## deviance    mu[1]    mu[2]    mu[3]    mu[4]       vv 
##     6448     6000     6000     6000     5800     6000
\end{verbatim}
\end{kframe}

{\centering \includegraphics[width=.4\textwidth]{figure/graphic-unnamed-chunk-191} 
\includegraphics[width=.4\textwidth]{figure/graphic-unnamed-chunk-192} 
\includegraphics[width=.4\textwidth]{figure/graphic-unnamed-chunk-193} 
\includegraphics[width=.4\textwidth]{figure/graphic-unnamed-chunk-194} 
\includegraphics[width=.4\textwidth]{figure/graphic-unnamed-chunk-195} 
\includegraphics[width=.4\textwidth]{figure/graphic-unnamed-chunk-196} 
\includegraphics[width=.4\textwidth]{figure/graphic-unnamed-chunk-197} 
\includegraphics[width=.4\textwidth]{figure/graphic-unnamed-chunk-198} 
\includegraphics[width=.4\textwidth]{figure/graphic-unnamed-chunk-199} 
\includegraphics[width=.4\textwidth]{figure/graphic-unnamed-chunk-1910} 
\includegraphics[width=.4\textwidth]{figure/graphic-unnamed-chunk-1911} 
\includegraphics[width=.4\textwidth]{figure/graphic-unnamed-chunk-1912} 
\includegraphics[width=.4\textwidth]{figure/graphic-unnamed-chunk-1913} 

}



\end{knitrout}


\begin{knitrout}\scriptsize
\definecolor{shadecolor}{rgb}{0.969, 0.969, 0.969}\color{fgcolor}\begin{kframe}
\begin{alltt}
mdl <- 'model \{
\hlcomment{    # Likelihood}
    \hlfunctioncall{for} (i in 1:N)\{
        response[i] ~ \hlfunctioncall{dnorm}(mu[tmt[i]], prec[tmt[i]])
    \}

\hlcomment{    # Create priors for each treatment}
    \hlfunctioncall{for}(i in 1:4)\{
        mu[i] ~ \hlfunctioncall{dnorm}(15,.0001)
        prec[i] <- 1/vv[i]
        vv[i] ~ \hlfunctioncall{dgamma}(1.1,.1)
    \}
\}
'
\hlfunctioncall{writeLines}(mdl, \hlstring{"code/ANOVAmodelNCvariance.jags"})
\end{alltt}
\end{kframe}
\end{knitrout}

\begin{knitrout}\scriptsize
\definecolor{shadecolor}{rgb}{0.969, 0.969, 0.969}\color{fgcolor}\begin{kframe}
\begin{alltt}
si
\end{alltt}


{\ttfamily\noindent\bfseries\textcolor{errorcolor}{\#\# Error: object 'si' not found}}\end{kframe}
\end{knitrout}


Devience is the -2*log likelihood

\subsection{Compare Means} % (fold)
\label{sub:compare_means}
We don't care about multiple test becasue we are not under the constraint of a null hypothisis
\begin{knitrout}\scriptsize
\definecolor{shadecolor}{rgb}{0.969, 0.969, 0.969}\color{fgcolor}\begin{kframe}
\begin{alltt}
diff12 <- sims[, 2] - sims[, 3]
\hlfunctioncall{mean}(dif12 > 0)
\end{alltt}


{\ttfamily\noindent\bfseries\textcolor{errorcolor}{\#\# Error: object 'dif12' not found}}\begin{alltt}
dif13 <- sims[, 2] - sims[, 4]
dif14 <- sims[, 2] - sims[, 5]
dif23 <- sims[, 3] - sims[, 4]
dif24 <- sims[, 3] - sims[, 5]
dif34 <- sims[, 4] - sims[, 5]
\hlfunctioncall{mean}(dif13 > 0)
\end{alltt}
\begin{verbatim}
## [1] 0.9958
\end{verbatim}
\begin{alltt}
\hlfunctioncall{mean}(dif14 > 0)
\end{alltt}
\begin{verbatim}
## [1] 0.9277
\end{verbatim}
\begin{alltt}
\hlfunctioncall{mean}(dif23 > 0)
\end{alltt}
\begin{verbatim}
## [1] 1
\end{verbatim}
\begin{alltt}
\hlfunctioncall{mean}(dif24 > 0)
\end{alltt}
\begin{verbatim}
## [1] 0.9995
\end{verbatim}
\begin{alltt}
\hlfunctioncall{mean}(dif34 > 0)
\end{alltt}
\begin{verbatim}
## [1] 0.08267
\end{verbatim}
\end{kframe}
\end{knitrout}

% subsection compare_means (end)

Two by three factorial design.
\begin{knitrout}\scriptsize
\definecolor{shadecolor}{rgb}{0.969, 0.969, 0.969}\color{fgcolor}\begin{kframe}
\begin{alltt}
twoway <- \hlfunctioncall{read.table}(\hlstring{"data/05twoway.dat"}, col.names = \hlfunctioncall{c}(\hlstring{"block"}, \hlstring{"seedType"}, \hlstring{"inoculate"}, 
    \hlstring{"yield"}))
\end{alltt}
\end{kframe}
\end{knitrout}

Using cell means method

Table should be rotated
\begin{tabular}{c|cc|c}
 & a & b & \\
 \hline
dea & $\mu_{11}$ & $\mu_{12}$ & $\mu_{1\cdot}$\\
con & $\mu_{21}$ & $\mu_{22}$ & $\mu_{2\cdot}$\\
liv & $\mu_{31}$ & $\mu_{22}$ & $\mu_{3\cdot}$\\
\hline
    & $\mu_{\cdot 1}$ & $\mu_{\cdot 2}$ & \\
\end{tabular}

Degrees of freedom is 5, one for type, two for inoculate, and two for interaction

\begin{knitrout}\scriptsize
\definecolor{shadecolor}{rgb}{0.969, 0.969, 0.969}\color{fgcolor}\begin{kframe}
\begin{alltt}
y <- \hlfunctioncall{as.numeric}(twoway$seedType)

mdl <- '
model \{
    \hlfunctioncall{for} (i in 1:24) \{
        yield[i] ~ \hlfunctioncall{dnorm}(mu[ncult[i],ninoc[i]], 1/vv)
    \}

    \hlfunctioncall{for} (i in 1:2) \{
        \hlfunctioncall{for} (j in 1:3) \{
            mu[i,j] ~ \hlfunctioncall{dnorm}(0, .000001)
        \}
    \}

    vv ~ \hlfunctioncall{dgamma}(1.5, .1)
\}
'

\hlfunctioncall{writeLines}(mdl, \hlstring{'code/CultModel.jags'})

yield <- twoway$yield
ncult <- \hlfunctioncall{as.numeric}(twoway$seedType)
ninoc <- \hlfunctioncall{as.numeric}(twoway$inoculate)

jags.data <- \hlfunctioncall{c}(\hlstring{'yield'}, \hlstring{'ncult'}, \hlstring{'ninoc'})
jags.params <- \hlfunctioncall{c}(\hlstring{'mu'},\hlstring{'vv'})

innits <- \hlfunctioncall{function}()\{\hlfunctioncall{list}(\hlstring{'mu'}= \hlfunctioncall{matrix}(0,2,3), \hlstring{'vv'}, 15)\}

cult1.jags <- \hlfunctioncall{jags}( data=jags.data,
\hlcomment{                    # inits=innits,}
                    parameters.to.save=jags.params,
                    model.file=\hlstring{'code/CultModel.jags'},
                    n.iter=12000, n.burnin=2000,
                    n.chains=1, n.thin=1)
\end{alltt}


{\ttfamily\noindent\itshape\textcolor{messagecolor}{\#\# module glm loaded}}\begin{verbatim}
## Compiling model graph
##    Resolving undeclared variables
##    Allocating nodes
##    Graph Size: 85
## 
## Initializing model
\end{verbatim}
\begin{alltt}
cult1.sim <- \hlfunctioncall{as.mcmc}(cult1.jags)
\end{alltt}
\end{kframe}
\end{knitrout}


Effects model and cell means model are two different ways to do analysis of variance. BYU was a pioneer of the cell means model.

\subsection{ANOVA Marginal Means} % (fold)
\label{sub:anova_marginal_means}

marginals are calculated as $\mu_{1\cdot} = \frac{\mu_{11} + \mu_{12} + \mu_{13}}{3}$
its just a mean of the means
\begin{knitrout}\scriptsize
\definecolor{shadecolor}{rgb}{0.969, 0.969, 0.969}\color{fgcolor}\begin{kframe}
\begin{alltt}
mu1dot <- (cult1.sim[, 2] + cult1.sim[, 4] + cult1.sim[, 6])/3
mu2dot <- (cult1.sim[, 3] + cult1.sim[, 5] + cult1.sim[, 7])/3

\hlfunctioncall{plot}(mu1dot - mu2dot)
\hlfunctioncall{mean}(mu2dot - mu1dot > 0)
\end{alltt}
\begin{verbatim}
## [1] 0.8171
\end{verbatim}
\begin{alltt}

mudot1 <- (cult1.sim[, 2] + cult1.sim[, 3])/2
mudot2 <- (cult1.sim[, 4] + cult1.sim[, 5])/2
mudot3 <- (cult1.sim[, 6] + cult1.sim[, 7])/2

\hlfunctioncall{mean}(mudot1 - mudot2 < 0)  \hlcomment{# Probability mu_dot2 is bigger than mu_dot1}
\end{alltt}
\begin{verbatim}
## [1] 0.9858
\end{verbatim}
\begin{alltt}
\hlfunctioncall{mean}(mudot3 - mudot2 > 0)  \hlcomment{# Probability mu_dot3 is bigger than mu_dot2}
\end{alltt}
\begin{verbatim}
## [1] 0.9991
\end{verbatim}
\end{kframe}

{\centering \includegraphics[width=.4\textwidth]{figure/graphic-unnamed-chunk-25} 

}



\end{knitrout}

% subsection anova_marginal_means (end)
\subsection{ANOVA Interactions} % (fold)
\label{sub:anova_interactions}
interactions are tested in four cell groups the formula is
\begin{align*}
    \mu_{11}-\mu_{21} = \mu_{12}-\mu_{22}\\
    \mu_{11}-\mu_{21} - \mu_{12}+\mu_{22} = 0
\end{align*}


\begin{knitrout}\scriptsize
\definecolor{shadecolor}{rgb}{0.969, 0.969, 0.969}\color{fgcolor}\begin{kframe}
\begin{alltt}
int1 <- cult1.sim[, 2] - cult1.sim[, 4] - cult1.sim[, 3] - cult1.sim[, 4]
int2 <- cult1.sim[, 4] - cult1.sim[, 6] - cult1.sim[, 5] + cult1.sim[, 7]
\end{alltt}
\end{kframe}
\end{knitrout}

% subsection anova_interactions (end)

Finally, compare to seed types that with the best inoculate, does it make sense to pay for the better seed or does the $\mu_{11}-\mu_{21} = \mu_{12}-\mu_{22}$ come from the inoculate
\begin{knitrout}\scriptsize
\definecolor{shadecolor}{rgb}{0.969, 0.969, 0.969}\color{fgcolor}\begin{kframe}
\begin{alltt}
se1 <- cult1.sim[, 6] - cult1.sim[, 7]
\hlfunctioncall{mean}(se1 < 0)
\end{alltt}
\begin{verbatim}
## [1] 0.6194
\end{verbatim}
\end{kframe}
\end{knitrout}

\subsection{Linear Algebra Solution} % (fold)
\label{sec:linear_algebra_solution}
% \begin{align*}
%     \begin{bmatrix}
%         27.4\\
%         29.7\\
%         34.5\\
%     \end{bmatrix}
%      =
%      \begin{bmatrix}
%      \end{bmatrix}
% \end{align*}

\[
\underline{\hat{\mu}} = (W' W)^{-1} W' \underline{y}
\]
Least Squares Solution
\begin{knitrout}\scriptsize
\definecolor{shadecolor}{rgb}{0.969, 0.969, 0.969}\color{fgcolor}\begin{kframe}
\begin{alltt}
w <- \hlfunctioncall{rbind}(\hlfunctioncall{diag}(1, 6), \hlfunctioncall{diag}(1, 6), \hlfunctioncall{diag}(1, 6), \hlfunctioncall{diag}(1, 6))
muhat <- \hlfunctioncall{solve}(\hlfunctioncall{t}(w) %*% w) %*% \hlfunctioncall{t}(w) %*% twoway$yield
\end{alltt}
\end{kframe}
\end{knitrout}


\begin{align*}
    y=W\mu\\
    y=WI\mu\\
    y= \underbrace{WA^{-1}}_{X} \underbrace{A\mu}_{\beta}
\end{align*}


\begin{align*}
\begin{bmatrix}
    \frac{1}{6} & \frac{1}{6}& \frac{1}{6}& \frac{1}{6}& \frac{1}{6}& \frac{1}{6}\\
    \frac{1}{6} & \frac{1}{6}& \frac{1}{6}& \frac{1}{6}& \frac{1}{6}& \frac{1}{6}\\
    \frac{1}{2} & -\frac{1}{2}& 0& \frac{1}{2}& -\frac{1}{2}& 0\\
    0&\frac{1}{2} & -\frac{1}{2}& 0&\frac{1}{2}& -\frac{1}{2}\\
    1 & -1 & 0 & -1 & 1 & 0\\
    0 & 1 & -1 & 0 & -1 & 1\\
\end{bmatrix}
\begin{bmatrix}
    \mu_{11}\\
    \mu_{12}\\
    \mu_{13}\\
    \mu_{21}\\
    \mu_{22}\\
    \mu_{23}\\
\end{bmatrix}
=
\begin{bmatrix}
    \mu_{\cdot\cdot}\\
    \mu_{1\cdot} - \mu_{2\cdot}\\
    \mu_{\cdot1} - \mu_{\cdot2}\\
    \mu_{\cdot2} - \mu_{\cdot3}\\
    \mu_{11}  - \mu_{12} - \mu_{21}+ \mu_{22}\\
    \mu_{12}  - \mu_{13} - \mu_{22}+ \mu_{23}\\
\end{bmatrix}
\end{align*}
% subsection linear_algebra_solution (end)
% section anova (end)
\renewcommand{\dateTaken}{February 12, 2013}
\daysep
\begin{knitrout}\scriptsize
\definecolor{shadecolor}{rgb}{0.969, 0.969, 0.969}\color{fgcolor}\begin{kframe}
\begin{alltt}
dev <- \hlfunctioncall{function}(x) \{
    fp <- \hlfunctioncall{length}(y) * \hlfunctioncall{log}(2* pi)
    sp <-
\}
\end{alltt}
\end{kframe}
\end{knitrout}

\renewcommand{\dateTaken}{February 14, 2013}
\daysep
\section{Analysis of Covariance ANCOVA} % (fold)
\label{sec:analysis_of_covariance_ancova}

Join Me 777-129-200

\begin{knitrout}\scriptsize
\definecolor{shadecolor}{rgb}{0.969, 0.969, 0.969}\color{fgcolor}\begin{kframe}
\begin{alltt}
ac <- \hlfunctioncall{read.table}(\hlstring{"data/06ancova.dat"}, header = TRUE)
\end{alltt}
\end{kframe}
\end{knitrout}


First thing you do for analysis is plot the data
\begin{knitrout}\scriptsize
\definecolor{shadecolor}{rgb}{0.969, 0.969, 0.969}\color{fgcolor}\begin{kframe}
\begin{alltt}
\hlfunctioncall{plot}(ac$speed, ac$scrap, pch = ac$lines)
\end{alltt}
\end{kframe}

{\centering \includegraphics[width=.4\textwidth]{figure/graphic-unnamed-chunk-31} 

}



\end{knitrout}


Let's try and come up with a model
\begin{align}
y_{i j} = \beta_{0} + \beta_{1}\cdot\text{Line} + \beta_{2}\cdot\text{Speed} + \beta_{3}\cdot \text{Line} \cdot \text{Speed}
\end{align}

Or we could do this
\begin{align*}
    y_{i j} = \beta_{0 i} + \beta_{1 i}\cdot\text{Speed}
\end{align*}

\begin{knitrout}\scriptsize
\definecolor{shadecolor}{rgb}{0.969, 0.969, 0.969}\color{fgcolor}\begin{kframe}
\begin{alltt}
mdl <- 'model \{
    \hlfunctioncall{for}(i in 1:27) \{
        scrap[i] ~ \hlfunctioncall{dnorm}(mu[i], prec);
        mu[i] <- b_0[line[i]] + b_1[line[i]]*speed[i];
    \}

    \hlfunctioncall{for} (i in 1:2) \{
        b_0[i] ~ \hlfunctioncall{dnorm}(30, .001);
        b_1[i] ~ \hlfunctioncall{dnorm}(0, .01);
    \}

    vr ~ \hlfunctioncall{dgamma}(1.5, .0125);
    prec <- 1/vr;
\}
'
\hlfunctioncall{writeLines}(mdl, \hlstring{'code/ANCOVAModel.jags'})
\end{alltt}
\end{kframe}
\end{knitrout}

prior for beta not covers 30*100

\begin{knitrout}\scriptsize
\definecolor{shadecolor}{rgb}{0.969, 0.969, 0.969}\color{fgcolor}\begin{kframe}
\begin{alltt}
speed <- ac$speed
line <- ac$line
scrap <- ac$scrap
jags.data <- \hlfunctioncall{c}(\hlstring{'speed'}, \hlstring{'line'}, \hlstring{'scrap'})
jags.params <- \hlfunctioncall{c}(\hlstring{'b_0'},\hlstring{'b_1'}, \hlstring{'vr'})
ancova.jags <- \hlfunctioncall{jags}( data=jags.data,
\hlcomment{                    # inits=innits,}
                    parameters.to.save=jags.params,
                    model.file=\hlstring{'code/ANCOVAModel.jags'},
                    n.iter=12000, n.burnin=2000,
                    n.chains=1, n.thin=1)
\end{alltt}


{\ttfamily\noindent\itshape\textcolor{messagecolor}{\#\# module glm loaded}}\begin{verbatim}
## Compiling model graph
##    Resolving undeclared variables
##    Allocating nodes
##    Graph Size: 148
## 
## Initializing model
\end{verbatim}
\begin{alltt}
ancova.jags
\end{alltt}
\begin{verbatim}
## Inference for Bugs model at "code/ANCOVAModel.jags", fit using jags,
##  1 chains, each with 12000 iterations (first 2000 discarded)
##  n.sims = 10000 iterations saved
##          mu.vect sd.vect    2.5%     25%     50%     75%   97.5%
## b_0[1]    81.738  15.570  50.441  71.518  81.938  92.185 111.585
## b_0[2]    13.553  16.417 -18.292   2.490  13.616  24.485  45.839
## b_1[1]     1.218   0.074   1.079   1.168   1.217   1.266   1.365
## b_1[2]     1.297   0.074   1.149   1.248   1.297   1.347   1.440
## vr       363.557  85.792 229.893 302.708 351.820 411.422 569.708
## deviance 241.252   3.179 236.996 238.870 240.626 242.949 249.002
## 
## DIC info (using the rule, pD = var(deviance)/2)
## pD = 5.1 and DIC = 246.3
## DIC is an estimate of expected predictive error (lower deviance is better).
\end{verbatim}
\end{kframe}
\end{knitrout}


Diagnostics
\begin{knitrout}\scriptsize
\definecolor{shadecolor}{rgb}{0.969, 0.969, 0.969}\color{fgcolor}\begin{kframe}
\begin{alltt}
ancova.sim <- \hlfunctioncall{as.mcmc}(ancova.jags)
\hlfunctioncall{plot}(ancova.sim, auto.layout = FALSE, ask = FALSE)
\hlfunctioncall{autocorr.plot}(ancova.sim, auto.layout = FALSE, ask = FALSE)
\hlfunctioncall{raftery.diag}(ancova.sim)
\end{alltt}
\begin{verbatim}
## 
## Quantile (q) = 0.025
## Accuracy (r) = +/- 0.005
## Probability (s) = 0.95 
##                                                 
##           Burn-in  Total Lower bound  Dependence
##           (M)      (N)   (Nmin)       factor (I)
##  b_0[1]   2        3650  3746         0.974     
##  b_0[2]   2        3680  3746         0.982     
##  b_1[1]   2        3680  3746         0.982     
##  b_1[2]   2        3834  3746         1.020     
##  deviance 2        3680  3746         0.982     
##  vr       2        3620  3746         0.966
\end{verbatim}
\end{kframe}

{\centering \includegraphics[width=.19\textwidth]{figure/graphic-ANCOVA_Diagnostics1} 
\includegraphics[width=.19\textwidth]{figure/graphic-ANCOVA_Diagnostics2} 
\includegraphics[width=.19\textwidth]{figure/graphic-ANCOVA_Diagnostics3} 
\includegraphics[width=.19\textwidth]{figure/graphic-ANCOVA_Diagnostics4} 
\includegraphics[width=.19\textwidth]{figure/graphic-ANCOVA_Diagnostics5} 
\includegraphics[width=.19\textwidth]{figure/graphic-ANCOVA_Diagnostics6} 
\includegraphics[width=.19\textwidth]{figure/graphic-ANCOVA_Diagnostics7} 
\includegraphics[width=.19\textwidth]{figure/graphic-ANCOVA_Diagnostics8} 
\includegraphics[width=.19\textwidth]{figure/graphic-ANCOVA_Diagnostics9} 
\includegraphics[width=.19\textwidth]{figure/graphic-ANCOVA_Diagnostics10} 
\includegraphics[width=.19\textwidth]{figure/graphic-ANCOVA_Diagnostics11} 
\includegraphics[width=.19\textwidth]{figure/graphic-ANCOVA_Diagnostics12} 
\includegraphics[width=.19\textwidth]{figure/graphic-ANCOVA_Diagnostics13} 
\includegraphics[width=.19\textwidth]{figure/graphic-ANCOVA_Diagnostics14} 
\includegraphics[width=.19\textwidth]{figure/graphic-ANCOVA_Diagnostics15} 
\includegraphics[width=.19\textwidth]{figure/graphic-ANCOVA_Diagnostics16} 
\includegraphics[width=.19\textwidth]{figure/graphic-ANCOVA_Diagnostics17} 

}



\end{knitrout}



Test the probability that the $\beta_{1\cdot}$ are different
\begin{knitrout}\scriptsize
\definecolor{shadecolor}{rgb}{0.969, 0.969, 0.969}\color{fgcolor}\begin{kframe}
\begin{alltt}
b11 <- ancova.sim[, 3]
b12 <- ancova.sim[, 4]
slopdif <- b11 - b12
\hlfunctioncall{mean}(slopdif < 0)
\end{alltt}
\begin{verbatim}
## [1] 0.7814
\end{verbatim}
\begin{alltt}
\hlfunctioncall{mean}(slopdif > 0)
\end{alltt}
\begin{verbatim}
## [1] 0.2186
\end{verbatim}
\end{kframe}
\end{knitrout}


Try a new model without the different $\beta_{1}$ then compare the DIC

\begin{knitrout}\scriptsize
\definecolor{shadecolor}{rgb}{0.969, 0.969, 0.969}\color{fgcolor}\begin{kframe}
\begin{alltt}
mdl <- 'model \{
    \hlfunctioncall{for}(i in 1:27) \{
        scrap[i] ~ \hlfunctioncall{dnorm}(mu[i], prec);
        mu[i] <- b_0[line[i]] + b_1*speed[i];
    \}

    b_1 ~ \hlfunctioncall{dnorm}(0, .01);
    \hlfunctioncall{for} (i in 1:2) \{
        b_0[i] ~ \hlfunctioncall{dnorm}(30, .001);
    \}

    vr ~ \hlfunctioncall{dgamma}(1.5, .0125);
    prec <- 1/vr;
\}
'
\hlfunctioncall{writeLines}(mdl, \hlstring{'code/ANCOVAModel_1.jags'})
\end{alltt}
\end{kframe}
\end{knitrout}

prior for beta not covers

\begin{knitrout}\scriptsize
\definecolor{shadecolor}{rgb}{0.969, 0.969, 0.969}\color{fgcolor}\begin{kframe}
\begin{alltt}
speed <- ac$speed
line <- ac$line
scrap <- ac$scrap
jags.data <- \hlfunctioncall{c}(\hlstring{'speed'}, \hlstring{'line'}, \hlstring{'scrap'})
jags.params <- \hlfunctioncall{c}(\hlstring{'b_0'},\hlstring{'b_1'}, \hlstring{'vr'})
ancova1.jags <- \hlfunctioncall{jags}( data=jags.data,
\hlcomment{                    # inits=innits,}
                    parameters.to.save=jags.params,
                    model.file=\hlstring{'code/ANCOVAModel_1.jags'},
                    n.iter=12000, n.burnin=2000,
                    n.chains=1, n.thin=1)
\end{alltt}
\begin{verbatim}
## Compiling model graph
##    Resolving undeclared variables
##    Allocating nodes
##    Graph Size: 142
## 
## Initializing model
\end{verbatim}
\begin{alltt}
ancova1.jags
\end{alltt}
\begin{verbatim}
## Inference for Bugs model at "code/ANCOVAModel_1.jags", fit using jags,
##  1 chains, each with 12000 iterations (first 2000 discarded)
##  n.sims = 10000 iterations saved
##          mu.vect sd.vect    2.5%     25%     50%     75%  97.5%
## b_0[1]    74.154  11.580  50.987  66.643  74.236  81.949  96.90
## b_0[2]    21.933  12.213  -2.044  13.621  22.005  30.055  45.95
## b_1        1.257   0.052   1.156   1.222   1.256   1.291   1.36
## vr       369.910  85.910 233.121 308.243 359.554 419.347 564.88
## deviance 242.053   2.761 238.636 240.016 241.464 243.406 248.97
## 
## DIC info (using the rule, pD = var(deviance)/2)
## pD = 3.8 and DIC = 245.9
## DIC is an estimate of expected predictive error (lower deviance is better).
\end{verbatim}
\end{kframe}
\end{knitrout}


Diagnostics
\begin{knitrout}\scriptsize
\definecolor{shadecolor}{rgb}{0.969, 0.969, 0.969}\color{fgcolor}\begin{kframe}
\begin{alltt}
ancova1.sim <- \hlfunctioncall{as.mcmc}(ancova1.jags)
\hlcomment{# plot(ancova.sim, auto.layout=FALSE, ask=FALSE) autocorr.plot(ancova.sim,}
\hlcomment{# auto.layout=FALSE, ask=FALSE) raftery.diag(ancova.sim)}
\end{alltt}
\end{kframe}
\end{knitrout}


Lets test differences in line by testing difference in intercept
\begin{knitrout}\scriptsize
\definecolor{shadecolor}{rgb}{0.969, 0.969, 0.969}\color{fgcolor}\begin{kframe}
\begin{alltt}
b_0_1 <- ancova1.sim[, 1]
b_0_2 <- ancova1.sim[, 2]
line_diff <- b_0_1 - b_0_2
\hlfunctioncall{mean}(line_diff < 0)
\end{alltt}
\begin{verbatim}
## [1] 0
\end{verbatim}
\begin{alltt}
\hlfunctioncall{mean}(line_diff > 0)
\end{alltt}
\begin{verbatim}
## [1] 1
\end{verbatim}
\begin{alltt}
\hlfunctioncall{plot}(\hlfunctioncall{density}(line_diff), col = \hlstring{"blue"})
\end{alltt}
\end{kframe}

{\centering \includegraphics[width=.4\textwidth]{figure/graphic-ANCOVA1_Line_Diff} 

}



\end{knitrout}


\begin{knitrout}\scriptsize
\definecolor{shadecolor}{rgb}{0.969, 0.969, 0.969}\color{fgcolor}\begin{kframe}
\begin{alltt}
\hlfunctioncall{plot}(ac$speed, ac$scrap, pch = ac$lines)
\hlfunctioncall{abline}(82.2, 1.22, col = \hlstring{"blue"})
\hlfunctioncall{abline}(13.7, 1.3, col = \hlstring{"red"})

b_0_1 <- ancova.sim[, 1]
b_0_2 <- ancova.sim[, 2]
b_1_1 <- ancova.sim[, 3]
b_1_2 <- ancova.sim[, 4]

y1_300 <- 300 * b_1_1 + b_0_1
y2_300 <- 300 * b_1_2 + b_0_2
diff_300 <- y1_300 - y2_300
\hlfunctioncall{mean}(diff_300 > 0)
\end{alltt}
\begin{verbatim}
## [1] 1
\end{verbatim}
\end{kframe}

{\centering \includegraphics[width=.4\textwidth]{figure/graphic-unnamed-chunk-37} 

}



\end{knitrout}

% section analysis_of_covariance_ancova (end)
\renewcommand{\dateTaken}{February 21, 2013}
\daysep
\begin{tabular}{ccccccc}
   &   &    & B  &    &    & \\
   &   &  1 &  2 &  3 &  4 & \\
   & 1 &  $\mu_{11}$ &  $\mu_{12}$ &  $\mu_{13}$ &  $mu_{14}$ &  $\mu_{1\bullet}$\\ %$\frac{\mu_{11} +  \mu_{12} +  \mu_{13} +  mu_{14}}{4}=\mu_{1\bullet}$\\
 A & 2 &  $\mu_{21}$ &  $\mu_{22}$ &  $\mu_{23}$ &  $mu_{24}$ &  $\mu_{2\bullet}$\\ %$\frac{\mu_{21} +  \mu_{22} +  \mu_{23} +  mu_{24}}{4}=\mu_{2\bullet}$\\
   & 3 &  $\mu_{31}$ &  $\mu_{32}$ &  $\mu_{33}$ &  $mu_{34}$ &  $\mu_{3\bullet}$\\ %$\frac{\mu_{31} +  \mu_{32} +  \mu_{33} +  mu_{34}}{4}=\mu_{3\bullet}$\\
   & 4 &  $\mu_{41}$ &  $\mu_{42}$ &  $\mu_{43}$ &  $mu_{44}$ &  $\mu_{4\bullet}$\\ %$\frac{\mu_{41} +  \mu_{42} +  \mu_{43} +  mu_{44}}{4}=\mu_{4\bullet}$\\
   % &   & $\frac{\mu_{11} +  \mu_{21} +  \mu_{31} +  mu_{41}}{4}=\mu_{\bullet 1}$
   %     & $\frac{\mu_{12} +  \mu_{22} +  \mu_{32} +  mu_{42}}{4}=\mu_{\bullet 2}$
   %     & $\frac{\mu_{13} +  \mu_{23} +  \mu_{33} +  mu_{43}}{4}=\mu_{\bullet 3}$
   %     & $\frac{\mu_{14} +  \mu_{24} +  \mu_{34} +  mu_{44}}{4}=\mu_{\bullet 4}$\\
\end{tabular}

\section{Multiple Sources of Variation} % (fold)
\label{sec:multiple_sources_of_variation}
\begin{align*}
    \underline{y}= X\underline{\beta} + \underline{\varepsilon}\\
    \varepsilon \sim \mathcal{N}(\underline{0}, \sigma^{2 \underline{T}})\\
    \underline{y} = X \underline{\beta} + Z \underline{u} + e\\
    e\sim \mathcal{N}(\underline{0}, R)\\
    \underline{u} \sim \mathcal{N}(\underline{0}, G)\\
\end{align*}

\begin{align*}
    \expv(Y)= \expv(X \beta + Z u + e) = \expv(X \beta) + \expv(Z u) + \expv(e)= X \beta\\
    \var(Y) = \var(X \beta + Z u + e) = \var( Z u + e)
\end{align*}
If we assume that $u$ and $e$ are independent then we can move forward easily

$V(A\underline{z})=A V(z) A'$
$A V(z) A'$ is a $p \times p$ matrix
\begin{align*}
    \var( Z u + e)= V(Z\underline{u}) + \var(\underline{e}) = Z \var(\underline{u}) Z' + V(\underline{e} = Z G Z' + R
\end{align*}
\renewcommand{\dateTaken}{February 26, 2013}
\daysep
Treatments:
\begin{tabular}{ccc}
Block 1 &  & Block 2\\
I ($y_{11}$) & & I ($y_{12}$)\\
II ($y_{21}$) & & II ($y_{22}$)\\
\end{tabular}

Model in matrix form (Equation~\ref{eq:mv_matrix_model})
\begin{align}\label{eq:mv_matrix_model}
    \begin{bmatrix}
        y_{11}\\
        y_{12}\\
        y_{21}\\
        y_{22}\\
    \end{bmatrix}
    =\begin{bmatrix}
        1 & 0\\
        1 & 0\\
        0 & 1\\
        0 & 1\\
    \end{bmatrix}
\begin{pmatrix}
    \mu_{1}\\
    \mu_{2}\\
\end{pmatrix} +
\begin{bmatrix}
    1 & 0\\
    0 & 1\\
    1 & 0\\
    0 & 1\\
\end{bmatrix}
\begin{pmatrix}
    u_{1}\\
    u_{2}\\
\end{pmatrix} +
\begin{bmatrix}
    e_{11}\\
    e_{12}\\
    e_{21}\\
    e_{22}\\
\end{bmatrix}
\end{align}

\begin{align*}
    e_{ij} \sim^{iid} N(0, \sigma^{2}_{e})\\
    u_{j} \sim^{iid} N(0,\sigma^{2}_{b}) \\
\end{align*}
\begin{align*}
    R &= \begin{bmatrix}
        \sigma^{2}_{e} & 0 & 0 & 0\\
        0 & \sigma^{2}_{e} & 0 & 0\\
        0 & 0 & \sigma^{2}_{e} & 0\\
        0 & 0 & 0 & \sigma^{2}_{e}\\
    \end{bmatrix}\\
    G &= \begin{pmatrix}
        \sigma^{2}_{b} & 0\\
        0 & \sigma^{2}_{b}
    \end{pmatrix}
\end{align*}


\begin{align*}
\begin{bmatrix}
        1 & 0\\
        1 & 0\\
        0 & 1\\
        0 & 1\\
\end{bmatrix}
\begin{pmatrix}
        \sigma^{2}_{b} & 0\\
        0 & \sigma^{2}_{b}
    \end{pmatrix}
    \begin{bmatrix}
        1 & 0 & 1 & 0\\
        0 & 1 & 0 & 1\\
    \end{bmatrix}
    +\begin{bmatrix}
        \sigma^{2}_{e} & 0 & 0 & 0\\
        0 & \sigma^{2}_{e} & 0 & 0\\
        0 & 0 & \sigma^{2}_{e} & 0\\
        0 & 0 & 0 & \sigma^{2}_{e}\\
    \end{bmatrix}\\
    Z G Z' + R = \begin{bmatrix}
        \sigma^{2}_{e} + \sigma^{2}_{b} & 0 & \sigma^{2}_{b} & 0\\
        0 & \sigma^{2}_{e} + \sigma^{2}_{b}  & 0 & \sigma^{2}_{b}\\
        \sigma^{2}_{b}  & 0 & \sigma^{2}_{e} + \sigma^{2}_{b} & 0 \\
        0 & \sigma^{2}_{b} & 0 & \sigma^{2}_{e} + \sigma^{2}_{b}\\
    \end{bmatrix}
\end{align*}
\begin{align*}
    \rho_{y_{11} y_{21}} = \frac{\sigma^{2}_{b}}{\sqrt{\sigma^{2}_{3} + \sigma^{2}_{b}}\sqrt{\sigma^{2}_{3} + \sigma^{2}_{b}}} =\frac{\sigma^{2}_{b}}{\sigma^{2}_{3} + \sigma^{2}_{b}}
\end{align*}
% section multiple_sources_of_variation (end)
\begin{knitrout}\scriptsize
\definecolor{shadecolor}{rgb}{0.969, 0.969, 0.969}\color{fgcolor}\begin{kframe}
\begin{alltt}
twoway <- \hlfunctioncall{read.table}(\hlstring{"data/05twoway.dat"}, col.names = \hlfunctioncall{c}(\hlstring{"block"}, \hlstring{"seedType"}, \hlstring{"inoculate"}, 
    \hlstring{"yield"}))
\end{alltt}
\end{kframe}
\end{knitrout}


First model with multiple souces of variance, s2blk is a hyper-prior, or a prior on a prior
\begin{knitrout}\scriptsize
\definecolor{shadecolor}{rgb}{0.969, 0.969, 0.969}\color{fgcolor}\begin{kframe}
\begin{alltt}
mdl <- ' model \{
    \hlfunctioncall{for} (i in 1:21) \{
        yield[i] ~ \hlfunctioncall{dnorm}(mu[i], 1/s2e)
        mu[i] <- aaron[cultn[i], inocn[i]] + u[block[i]]
    \}

    \hlfunctioncall{for} (i in 1:2)\{
        \hlfunctioncall{for} (j in 1:3) \{
            aaron[i,j] ~ \hlfunctioncall{dnorm}(30, .001);

        \}
    \}

    \hlfunctioncall{for} (i in 1:4) \{
        u[i] ~ \hlfunctioncall{dnorm}(0, 1/s2blk)
    \}

    s2e ~ \hlfunctioncall{dgamma}(1.5, .1);
    s2blk ~ \hlfunctioncall{dgamma}(1.5, .1);
\}
'
\hlfunctioncall{writeLines}(mdl, \hlstring{'code/MixedModels.jags'})
\end{alltt}
\end{kframe}
\end{knitrout}


\begin{knitrout}\scriptsize
\definecolor{shadecolor}{rgb}{0.969, 0.969, 0.969}\color{fgcolor}\begin{kframe}
\begin{alltt}
yield <- twoway[, 4]
cultn <- \hlfunctioncall{as.numeric}(twoway$seedType)
inocn <- \hlfunctioncall{as.numeric}(twoway$inoculate)
block <- twoway[, 1]

jags.data <- \hlfunctioncall{c}(\hlstring{"yield"}, \hlstring{"cultn"}, \hlstring{"inocn"}, \hlstring{"block"})
jags.params <- \hlfunctioncall{c}(\hlstring{"aaron"}, \hlstring{"s2e"}, \hlstring{"s2blk"})
\hlfunctioncall{set.seed}(343)
mixedm.jags <- \hlfunctioncall{jags}(data = jags.data, parameters.to.save = jags.params, model.file = \hlstring{"code/MixedModels.jags"}, 
    n.iter = 12000, n.burnin = 2000, n.chains = 1, n.thin = 1)
\end{alltt}


{\ttfamily\noindent\itshape\textcolor{messagecolor}{\#\# module glm loaded}}\begin{verbatim}
## Compiling model graph
##    Resolving undeclared variables
##    Allocating nodes
##    Graph Size: 137
## 
## Initializing model
\end{verbatim}
\end{kframe}
\end{knitrout}


Difference between 1 and 2
\begin{knitrout}\scriptsize
\definecolor{shadecolor}{rgb}{0.969, 0.969, 0.969}\color{fgcolor}\begin{kframe}
\begin{alltt}
mixedm.sim <- \hlfunctioncall{as.mcmc}(mixedm.jags)
cultivarA <- (mixedm.sim[, 1] + mixedm.sim[, 3] + mixedm.sim[, 5])/3
cultivarB <- (mixedm.sim[, 2] + mixedm.sim[, 4] + mixedm.sim[, 6])/3

\hlfunctioncall{mean}(cultivarA)
\end{alltt}
\begin{verbatim}
## [1] 30.09
\end{verbatim}
\begin{alltt}
\hlfunctioncall{mean}(cultivarB)
\end{alltt}
\begin{verbatim}
## [1] 31.13
\end{verbatim}
\begin{alltt}
\hlfunctioncall{plot}(\hlfunctioncall{density}(cultivarA - cultivarB))
\end{alltt}
\end{kframe}

{\centering \includegraphics[width=.4\textwidth]{figure/graphic-unnamed-chunk-39} 

}



\end{knitrout}

\renewcommand{\dateTaken}{February 28, 2013}
\daysep
Simpler Mixed Model
\begin{knitrout}\scriptsize
\definecolor{shadecolor}{rgb}{0.969, 0.969, 0.969}\color{fgcolor}\begin{kframe}
\begin{alltt}
bond <- \hlfunctioncall{read.table}(\hlstring{"data/07mixedmods.dat"}, header = TRUE)
metn <- \hlfunctioncall{rep}(1:3, 21/3)
bond <- \hlfunctioncall{cbind}(bond, metn)
\end{alltt}
\end{kframe}
\end{knitrout}

\begin{knitrout}\scriptsize
\definecolor{shadecolor}{rgb}{0.969, 0.969, 0.969}\color{fgcolor}\begin{kframe}
\begin{alltt}
mdl <- \hlstring{" model\{\textbackslash{}\hlfunctioncall{nfor} (i in 1:21)\{\textbackslash{}npressure[i] ~ \hlfunctioncall{dnorm}(mu[i], 1/s2error);\textbackslash{}nmu[i] <- gamma[metn[i]] + u[ingot[i]]\textbackslash{}n\}\textbackslash{}\hlfunctioncall{ninvisible}(\textbackslash{}"}.BeGiN_TiDy_IdEnTiFiEr_HaHaHa.HaHaHa_EnD_TiDy_IdEnTiFiEr\textbackslash{}\hlstring{")\textbackslash{}\hlfunctioncall{nfor} (i in 1:3) \{\textbackslash{}ngamma[i] ~ \hlfunctioncall{dnorm}(75, 0.001)\textbackslash{}n\}\textbackslash{}\hlfunctioncall{ninvisible}(\textbackslash{}"}.BeGiN_TiDy_IdEnTiFiEr_HaHaHa.HaHaHa_EnD_TiDy_IdEnTiFiEr\textbackslash{}\hlstring{")\textbackslash{}\hlfunctioncall{nfor} (i in 1:7) \{\textbackslash{}nu[i] ~ \hlfunctioncall{dnorm}(0, 1/s2ing)\textbackslash{}n\}\textbackslash{}\hlfunctioncall{ninvisible}(\textbackslash{}"}.BeGiN_TiDy_IdEnTiFiEr_HaHaHa.HaHaHa_EnD_TiDy_IdEnTiFiEr\textbackslash{}\hlstring{")\textbackslash{}ns2error ~ \hlfunctioncall{dgamma}(1.5, 0.1);\textbackslash{}ns2ing   ~ \hlfunctioncall{dgamma}(1.5, 0.1);\textbackslash{}n\}\textbackslash{}n"}
\hlfunctioncall{writeLines}(mdl, \hlstring{"code/SimpleMixedModel.jags"})
\end{alltt}
\end{kframe}
\end{knitrout}

\begin{knitrout}\scriptsize
\definecolor{shadecolor}{rgb}{0.969, 0.969, 0.969}\color{fgcolor}\begin{kframe}
\begin{alltt}
metn <- bond[, 4]
ingot <- bond[, 1]
pressure <- bond[, 3]

jags.data <- \hlfunctioncall{c}(\hlstring{"metn"}, \hlstring{"ingot"}, \hlstring{"pressure"})
jags.params <- \hlfunctioncall{c}(\hlstring{"gamma"}, \hlstring{"s2error"}, \hlstring{"s2ing"})

bond.jags <- \hlfunctioncall{jags}(data = jags.data, parameters.to.save = jags.params, model.file = \hlstring{"code/SimpleMixedModel.jags"}, 
    n.iter = 12000, n.burnin = 2000, n.chains = 1, n.thin = 1)
\end{alltt}


{\ttfamily\noindent\itshape\textcolor{messagecolor}{\#\# module glm loaded}}

{\ttfamily\noindent\bfseries\textcolor{errorcolor}{\#\# Error: \\\#\# Error parsing model file:\\\#\# syntax error on line 6 near """}}\begin{alltt}
bond.sim <- \hlfunctioncall{as.mcmc}(bond.jags)
\end{alltt}


{\ttfamily\noindent\bfseries\textcolor{errorcolor}{\#\# Error: object 'bond.jags' not found}}\end{kframe}
\end{knitrout}


ICC or interclass corelation is a measure of reliability of messurements
\begin{align}
    ICC = \frac{\sigma^{2}_{i}}{\sigma^{2}_{e}+\sigma^{2}_{i}}
\end{align}
\begin{knitrout}\scriptsize
\definecolor{shadecolor}{rgb}{0.969, 0.969, 0.969}\color{fgcolor}\begin{kframe}
\begin{alltt}
icc <- bond.sim[, 5]/(bond.sim[, 6] + bond.sim[, 5])
\end{alltt}


{\ttfamily\noindent\bfseries\textcolor{errorcolor}{\#\# Error: object 'bond.sim' not found}}\begin{alltt}
\hlfunctioncall{plot}(\hlfunctioncall{density}(icc))
\end{alltt}


{\ttfamily\noindent\bfseries\textcolor{errorcolor}{\#\# Error: object 'icc' not found}}\end{kframe}
\end{knitrout}


plot something
\begin{knitrout}\scriptsize
\definecolor{shadecolor}{rgb}{0.969, 0.969, 0.969}\color{fgcolor}\begin{kframe}
\begin{alltt}
indif <- bond.sim[, 3] - bond.sim[, 2]
\end{alltt}


{\ttfamily\noindent\bfseries\textcolor{errorcolor}{\#\# Error: object 'bond.sim' not found}}\begin{alltt}
icdif <- bond.sim[, 3] - bond.sim[, 4]
\end{alltt}


{\ttfamily\noindent\bfseries\textcolor{errorcolor}{\#\# Error: object 'bond.sim' not found}}\end{kframe}
\end{knitrout}


\textbf{Back to Complex}
\begin{tabular}{ccccc}
 &  &  & inoc & \\
 &  & con & dea & liv \\
 cult & a & $\alpha_{1 1}$ & $\alpha_{1 2}$ & $\alpha_{1 3}$\\
 & b & $\alpha_{2 1}$ & $\alpha_{2 2}$ & $\alpha_{2 3}$\\
\end{tabular}
\begin{knitrout}\scriptsize
\definecolor{shadecolor}{rgb}{0.969, 0.969, 0.969}\color{fgcolor}\begin{kframe}
\begin{alltt}
int1 <- mixedm.sim[, 2] - mixedm.sim[, 4] - mixedm.sim[, 3] + mixedm.sim[, 5]
int2 <- mixedm.sim[, 4] - mixedm.sim[, 6] - mixedm.sim[, 5] + mixedm.sim[, 7]
\hlfunctioncall{plot}(\hlfunctioncall{density}(int1))
\hlfunctioncall{plot}(\hlfunctioncall{density}(int2))
\end{alltt}
\end{kframe}

{\centering \includegraphics[width=.4\textwidth]{figure/graphic-unnamed-chunk-451} 
\includegraphics[width=.4\textwidth]{figure/graphic-unnamed-chunk-452} 

}



\end{knitrout}

Let's say that there's no interaction, because

\renewcommand{\dateTaken}{March  5, 2013}
\daysep

\section{Random Coefficients Model} % (fold)
\label{sec:random_coefficients_model}
Genetically identical seeds we treat as subjects

Sixty observations, but only ten wheat types, so there are really somewhere in the middle

\begin{align*}
    \text{yield}_{i j} = \beta_{0} + \beta_{1}\cdot \text{moisture}_{j} + u1_{i} + u2_{i} \cdot \text{moisture}_{j} + \text{error}_{ij}
\end{align*}
so the deviation $u$ has both a slope and an intercept.

% \begin{align*}
%     y=X \beta + Z u + e
% \end{align*}
\begin{align*}
    y = \begin{bmatrix}
        1 & 10\\
        1 & 17\\
        1 & \vdots\\
        \vdots & \ddots\\
    \end{bmatrix}
    \begin{pmatrix}
        \beta_{0}\\
        \beta_{1}
    \end{pmatrix}
    \begin{bmatrix}
        1 & 10 & 0 & 0 & 0 & 0 & \ldots \\
        1 & 57 & 0 & 0 & 0 & 0 & \ldots \\
        \vdots & \vdots & \vdots & \vdots & \vdots & \vdots & \ldots \\
        1 & 40 & 0 & 0 & 0 & 0 & \ldots \\
        0 & 0 & 1 & 16 & 0 & 0 & \ldots \\
        \vdots & \vdots & \vdots & \vdots & \vdots & \vdots & \ldots \\
        0 & 0 & 0 & 0 & 1 & 39 & \ldots \\
        \vdots & \vdots & \vdots & \vdots & \vdots & \vdots & \ddots
    \end{bmatrix}
\begin{pmatrix}
    u0_{1}\\
    u1_{1}\\
    u0_{2}\\
    u1_{2}\\
    u0_{3}\\
    u1_{3}\\
    u0_{4}\\
    u1_{4}\\
    u0_{5}\\
    u1_{5}\\
    u0_{6}\\
    u1_{6}\\
\end{pmatrix}
\end{align*}
\begin{align*}
    G=\begin{bmatrix}
        \sigma^{2}_{int} & 0                & 0                & 0 & \ldots \\
        0                & \sigma^{2}_{slp} & 0                & 0 & \ldots \\
        0                & 0                & \sigma^{2}_{int} & 0 & \ldots \\
        0                & 0                & 0                & \sigma^{2}_{slp} & \ldots \\
        \vdots & \vdots & \vdots & \vdots & \ddots
    \end{bmatrix}
\end{align*}

\begin{knitrout}\scriptsize
\definecolor{shadecolor}{rgb}{0.969, 0.969, 0.969}\color{fgcolor}\begin{kframe}
\begin{alltt}
wheat <- \hlfunctioncall{read.table}(\hlstring{"data/10randomcoef.dat"}, header=TRUE)
yield <- wheat$yield
variety <- wheat$variety
moisture <- wheat$moisture
mdl <- ' model \{
    \hlfunctioncall{for} (i in 1:60) \{
        yield[i] ~ \hlfunctioncall{dnorm}(mu[i], 1/s2err)
        mu[i] <- b0 + b1 * moisture[i] + u0[variety[i]] + u1[variety[i]] * moisture[i]
    \}

    b0 ~ \hlfunctioncall{dnorm}(30, 0.001)
    b1 ~ \hlfunctioncall{dnorm}(0, 0.1)

    \hlfunctioncall{for} (i in 1:10) \{
        u0[i] ~ \hlfunctioncall{dnorm}(0, 1/s2int)
        u1[i] ~ \hlfunctioncall{dnorm}(0, 1/s2slp)
    \}

    s2err ~ \hlfunctioncall{dgamma}(1.1, .5)
    s2int ~ \hlfunctioncall{dgamma}(1.1, .1)
    s2slp ~ \hlfunctioncall{dgamma}(1.1, 2)
\}
'
\hlfunctioncall{writeLines}(mdl, \hlstring{'code/RandomCoeff.jags'})
\end{alltt}
\end{kframe}
\end{knitrout}


\begin{knitrout}\scriptsize
\definecolor{shadecolor}{rgb}{0.969, 0.969, 0.969}\color{fgcolor}\begin{kframe}
\begin{alltt}
jags.data <- \hlfunctioncall{c}(\hlstring{"yield"}, \hlstring{"variety"}, \hlstring{"moisture"})
jags.params <- \hlfunctioncall{c}(\hlstring{"b0"}, \hlstring{"b1"}, \hlstring{"u0"}, \hlstring{"u1"}, \hlstring{"s2err"}, \hlstring{"s2int"}, \hlstring{"s2slp"})

rc1.jags <- \hlfunctioncall{jags}(data = jags.data, parameters.to.save = jags.params, model.file = \hlstring{"code/RandomCoeff.jags"}, 
    n.iter = 12000, n.burnin = 2000, n.chains = 1, n.thin = 1)
\end{alltt}


{\ttfamily\noindent\itshape\textcolor{messagecolor}{\#\# module glm loaded}}\begin{verbatim}
## Compiling model graph
##    Resolving undeclared variables
##    Allocating nodes
##    Graph Size: 366
## 
## Initializing model
\end{verbatim}
\begin{alltt}
rc1.jags
\end{alltt}
\begin{verbatim}
## Inference for Bugs model at "code/RandomCoeff.jags", fit using jags,
##  1 chains, each with 12000 iterations (first 2000 discarded)
##  n.sims = 10000 iterations saved
##          mu.vect sd.vect    2.5%     25%     50%     75%   97.5%
## b0        33.406   1.355  30.710  32.525  33.397  34.263  36.139
## b1         0.662   0.022   0.617   0.648   0.662   0.675   0.704
## s2err      0.389   0.092   0.248   0.322   0.378   0.442   0.603
## s2int     17.953   7.124   8.213  12.823  16.548  21.613  36.010
## s2slp      0.004   0.003   0.001   0.002   0.003   0.005   0.013
## u0[1]      0.958   1.472  -1.995  -0.016   0.964   1.925   3.839
## u0[2]     -2.114   1.518  -5.174  -3.128  -2.109  -1.107   0.894
## u0[3]     -0.371   1.501  -3.410  -1.352  -0.376   0.639   2.534
## u0[4]      0.716   1.447  -2.145  -0.225   0.713   1.656   3.551
## u0[5]      1.097   1.664  -2.225  -0.001   1.102   2.217   4.370
## u0[6]      4.602   1.434   1.760   3.684   4.604   5.529   7.451
## u0[7]    -10.563   1.420 -13.406 -11.479 -10.560  -9.637  -7.808
## u0[8]      2.380   1.449  -0.466   1.433   2.365   3.350   5.277
## u0[9]     -0.162   1.492  -3.148  -1.131  -0.170   0.815   2.783
## u0[10]     3.573   1.840  -0.041   2.339   3.551   4.809   7.153
## u1[1]     -0.049   0.026  -0.099  -0.065  -0.049  -0.032   0.002
## u1[2]     -0.072   0.033  -0.141  -0.093  -0.071  -0.050  -0.009
## u1[3]      0.067   0.028   0.016   0.049   0.066   0.085   0.124
## u1[4]     -0.023   0.028  -0.080  -0.041  -0.023  -0.005   0.033
## u1[5]     -0.019   0.030  -0.079  -0.038  -0.019   0.001   0.040
## u1[6]      0.026   0.025  -0.022   0.010   0.025   0.041   0.075
## u1[7]      0.051   0.027  -0.001   0.033   0.050   0.068   0.107
## u1[8]      0.024   0.027  -0.028   0.007   0.023   0.041   0.079
## u1[9]      0.024   0.027  -0.028   0.006   0.024   0.041   0.077
## u1[10]    -0.029   0.033  -0.095  -0.050  -0.029  -0.008   0.034
## deviance 110.389   8.401  96.576 104.372 109.459 115.604 128.815
## 
## DIC info (using the rule, pD = var(deviance)/2)
## pD = 35.3 and DIC = 145.7
## DIC is an estimate of expected predictive error (lower deviance is better).
\end{verbatim}
\begin{alltt}
rc1.sim <- \hlfunctioncall{as.mcmc}(rc1.jags)
\hlfunctioncall{plot}(rc1.sim, auto.layout = FALSE, ask = FALSE)
\end{alltt}
\end{kframe}

{\centering \includegraphics[width=.16\textwidth]{figure/graphic-RandomCoeffSim1} 
\includegraphics[width=.16\textwidth]{figure/graphic-RandomCoeffSim2} 
\includegraphics[width=.16\textwidth]{figure/graphic-RandomCoeffSim3} 
\includegraphics[width=.16\textwidth]{figure/graphic-RandomCoeffSim4} 
\includegraphics[width=.16\textwidth]{figure/graphic-RandomCoeffSim5} 
\includegraphics[width=.16\textwidth]{figure/graphic-RandomCoeffSim6} 
\includegraphics[width=.16\textwidth]{figure/graphic-RandomCoeffSim7} 
\includegraphics[width=.16\textwidth]{figure/graphic-RandomCoeffSim8} 
\includegraphics[width=.16\textwidth]{figure/graphic-RandomCoeffSim9} 
\includegraphics[width=.16\textwidth]{figure/graphic-RandomCoeffSim10} 
\includegraphics[width=.16\textwidth]{figure/graphic-RandomCoeffSim11} 
\includegraphics[width=.16\textwidth]{figure/graphic-RandomCoeffSim12} 
\includegraphics[width=.16\textwidth]{figure/graphic-RandomCoeffSim13} 
\includegraphics[width=.16\textwidth]{figure/graphic-RandomCoeffSim14} 
\includegraphics[width=.16\textwidth]{figure/graphic-RandomCoeffSim15} 
\includegraphics[width=.16\textwidth]{figure/graphic-RandomCoeffSim16} 
\includegraphics[width=.16\textwidth]{figure/graphic-RandomCoeffSim17} 
\includegraphics[width=.16\textwidth]{figure/graphic-RandomCoeffSim18} 
\includegraphics[width=.16\textwidth]{figure/graphic-RandomCoeffSim19} 
\includegraphics[width=.16\textwidth]{figure/graphic-RandomCoeffSim20} 
\includegraphics[width=.16\textwidth]{figure/graphic-RandomCoeffSim21} 
\includegraphics[width=.16\textwidth]{figure/graphic-RandomCoeffSim22} 
\includegraphics[width=.16\textwidth]{figure/graphic-RandomCoeffSim23} 
\includegraphics[width=.16\textwidth]{figure/graphic-RandomCoeffSim24} 
\includegraphics[width=.16\textwidth]{figure/graphic-RandomCoeffSim25} 
\includegraphics[width=.16\textwidth]{figure/graphic-RandomCoeffSim26} 
\includegraphics[width=.16\textwidth]{figure/graphic-RandomCoeffSim27} 
\includegraphics[width=.16\textwidth]{figure/graphic-RandomCoeffSim28} 
\includegraphics[width=.16\textwidth]{figure/graphic-RandomCoeffSim29} 
\includegraphics[width=.16\textwidth]{figure/graphic-RandomCoeffSim30} 
\includegraphics[width=.16\textwidth]{figure/graphic-RandomCoeffSim31} 
\includegraphics[width=.16\textwidth]{figure/graphic-RandomCoeffSim32} 
\includegraphics[width=.16\textwidth]{figure/graphic-RandomCoeffSim33} 
\includegraphics[width=.16\textwidth]{figure/graphic-RandomCoeffSim34} 
\includegraphics[width=.16\textwidth]{figure/graphic-RandomCoeffSim35} 
\includegraphics[width=.16\textwidth]{figure/graphic-RandomCoeffSim36} 
\includegraphics[width=.16\textwidth]{figure/graphic-RandomCoeffSim37} 
\includegraphics[width=.16\textwidth]{figure/graphic-RandomCoeffSim38} 
\includegraphics[width=.16\textwidth]{figure/graphic-RandomCoeffSim39} 
\includegraphics[width=.16\textwidth]{figure/graphic-RandomCoeffSim40} 
\includegraphics[width=.16\textwidth]{figure/graphic-RandomCoeffSim41} 
\includegraphics[width=.16\textwidth]{figure/graphic-RandomCoeffSim42} 
\includegraphics[width=.16\textwidth]{figure/graphic-RandomCoeffSim43} 
\includegraphics[width=.16\textwidth]{figure/graphic-RandomCoeffSim44} 
\includegraphics[width=.16\textwidth]{figure/graphic-RandomCoeffSim45} 
\includegraphics[width=.16\textwidth]{figure/graphic-RandomCoeffSim46} 
\includegraphics[width=.16\textwidth]{figure/graphic-RandomCoeffSim47} 
\includegraphics[width=.16\textwidth]{figure/graphic-RandomCoeffSim48} 
\includegraphics[width=.16\textwidth]{figure/graphic-RandomCoeffSim49} 
\includegraphics[width=.16\textwidth]{figure/graphic-RandomCoeffSim50} 
\includegraphics[width=.16\textwidth]{figure/graphic-RandomCoeffSim51} 
\includegraphics[width=.16\textwidth]{figure/graphic-RandomCoeffSim52} 

}



\end{knitrout}


Find the variance of the $y$
\begin{knitrout}\scriptsize
\definecolor{shadecolor}{rgb}{0.969, 0.969, 0.969}\color{fgcolor}\begin{kframe}
\begin{alltt}
g <- \hlfunctioncall{rbind}(\hlfunctioncall{cbind}(18.645, 0), \hlfunctioncall{cbind}(0, 0.005))
r <- \hlfunctioncall{diag}(0.392, 60)
z <- \hlfunctioncall{model.matrix}(~-1 + \hlfunctioncall{as.factor}(variety) + \hlfunctioncall{as.factor}(variety):moisture, wheat)
Z <- z[, \hlfunctioncall{c}(1, 11, 2, 12, 3, 13, 4, 14, 5, 15, 6, 16, 7, 17, 8, 18, 9, 19, 10, 20)]  \hlcomment{# need to rearrange so that the intercepts and slopes are interleaved}
G <- \hlfunctioncall{kronecker}(\hlfunctioncall{diag}(1, 10), g)
V <- Z %*% G %*% \hlfunctioncall{t}(Z) + r
\end{alltt}
\end{kframe}
\end{knitrout}

\renewcommand{\dateTaken}{March  7, 2013}
\daysep

To get the slope or intercept for the particular variety, add the fixed effect and the random effect for the variety.
% section random_coefficients_model (end)
\section{Hierarchical Models} % (fold)
\label{sec:hierarchical_models}
$y=\mu_{0}+\mu_{s}\text{moisture} + e$

$\mu_{0}~\mathcal{N}(\mu_{0}, \sigma^{2}_{0})$, $\mu_{s}\sim\mathcal{N}(\mu,\sigma^{2}_{s})$, $e\sim(0,\sigma^{2}_{e}$

\begin{knitrout}\scriptsize
\definecolor{shadecolor}{rgb}{0.969, 0.969, 0.969}\color{fgcolor}\begin{kframe}
\begin{alltt}
wheat <- \hlfunctioncall{read.table}(\hlstring{"data/10randomcoef.dat"}, header=TRUE)
yield <- wheat$yield
variety <- wheat$variety
moisture <- wheat$moisture
mdl <- ' model \{
    \hlfunctioncall{for} (i in 1:60) \{
        yield[i] ~ \hlfunctioncall{dnorm}(mu[i], 1/s2err)
        mu[i] <- u0[variety[i]] + u1[variety[i]] * moisture[i]
    \}

    b0 ~ \hlfunctioncall{dnorm}(30, 0.001)
    b1 ~ \hlfunctioncall{dnorm}(0, 0.1)

    \hlfunctioncall{for} (i in 1:10) \{
        u0[i] ~ \hlfunctioncall{dnorm}(beta0, 1/s2int)
        u1[i] ~ \hlfunctioncall{dnorm}(beta1, 1/s2slp)
    \}

    beta0 ~ \hlfunctioncall{dnorm}(30, 0.001)
    beta1 ~ \hlfunctioncall{dnorm}(0, 0.1)

    s2err ~ \hlfunctioncall{dgamma}(1.1, .5)
    s2int ~ \hlfunctioncall{dgamma}(1.1, .1)
    s2slp ~ \hlfunctioncall{dgamma}(1.1, 2)
\}
'
\hlfunctioncall{writeLines}(mdl, \hlstring{'code/Hierarchical.jags'})
\end{alltt}
\end{kframe}
\end{knitrout}


\begin{knitrout}\scriptsize
\definecolor{shadecolor}{rgb}{0.969, 0.969, 0.969}\color{fgcolor}\begin{kframe}
\begin{alltt}
jags.data <- \hlfunctioncall{c}(\hlstring{"yield"}, \hlstring{"variety"}, \hlstring{"moisture"})
jags.params <- \hlfunctioncall{c}(\hlstring{"beta0"}, \hlstring{"beta1"}, \hlstring{"u0"}, \hlstring{"u1"}, \hlstring{"s2err"}, \hlstring{"s2int"}, \hlstring{"s2slp"})

hier.jags <- \hlfunctioncall{jags}(data = jags.data, parameters.to.save = jags.params, model.file = \hlstring{"code/Hierarchical.jags"}, 
    n.iter = 12000, n.burnin = 2000, n.chains = 1, n.thin = 1)
\end{alltt}


{\ttfamily\noindent\itshape\color{messagecolor}{\#\# module glm loaded}}\begin{verbatim}
## Compiling model graph
##    Resolving undeclared variables
##    Allocating nodes
##    Graph Size: 334
## 
## Initializing model
\end{verbatim}
\begin{alltt}
hier.jags
\end{alltt}
\begin{verbatim}
## Inference for Bugs model at "code/Hierarchical.jags", fit using jags,
##  1 chains, each with 12000 iterations (first 2000 discarded)
##  n.sims = 10000 iterations saved
##          mu.vect sd.vect   2.5%     25%     50%     75%   97.5%
## beta0     33.422   1.366 30.708  32.548  33.423  34.308  36.162
## beta1      0.661   0.023  0.613   0.648   0.662   0.675   0.706
## s2err      0.391   0.094  0.247   0.324   0.379   0.444   0.613
## s2int     18.019   7.236  8.118  12.861  16.541  21.633  35.702
## s2slp      0.005   0.004  0.001   0.002   0.004   0.006   0.015
## u0[1]     34.367   0.689 33.034  33.905  34.362  34.826  35.724
## u0[2]     31.309   0.794 29.711  30.783  31.322  31.840  32.870
## u0[3]     33.022   0.706 31.659  32.538  33.019  33.497  34.405
## u0[4]     34.126   0.592 32.972  33.725  34.129  34.518  35.286
## u0[5]     34.508   1.114 32.321  33.749  34.511  35.256  36.712
## u0[6]     38.019   0.539 36.958  37.661  38.017  38.376  39.098
## u0[7]     22.834   0.499 21.876  22.506  22.821  23.160  23.828
## u0[8]     35.783   0.599 34.601  35.388  35.780  36.178  36.983
## u0[9]     33.254   0.704 31.901  32.778  33.251  33.717  34.645
## u0[10]    36.978   1.391 34.238  36.047  36.969  37.916  39.672
## u1[1]      0.613   0.016  0.581   0.603   0.613   0.623   0.644
## u1[2]      0.589   0.028  0.533   0.570   0.589   0.608   0.645
## u1[3]      0.729   0.018  0.694   0.717   0.729   0.741   0.764
## u1[4]      0.638   0.021  0.598   0.625   0.638   0.652   0.679
## u1[5]      0.643   0.023  0.596   0.627   0.642   0.659   0.688
## u1[6]      0.687   0.013  0.661   0.678   0.687   0.696   0.713
## u1[7]      0.713   0.018  0.678   0.701   0.713   0.725   0.748
## u1[8]      0.686   0.017  0.652   0.675   0.686   0.697   0.719
## u1[9]      0.685   0.018  0.650   0.674   0.685   0.697   0.720
## u1[10]     0.632   0.027  0.579   0.614   0.633   0.650   0.686
## deviance 110.551   8.430 96.364 104.516 109.746 115.607 129.397
## 
## DIC info (using the rule, pD = var(deviance)/2)
## pD = 35.5 and DIC = 146.1
## DIC is an estimate of expected predictive error (lower deviance is better).
\end{verbatim}
\begin{alltt}
hier.sim <- \hlfunctioncall{as.mcmc}(hier.jags)
\hlfunctioncall{plot}(hier.sim, auto.layout = inter_plots, ask = inter_plots)
\end{alltt}
\end{kframe}

{\centering \includegraphics[width=.16\textwidth]{figure/graphic-HierarchicalSim1} 
\includegraphics[width=.16\textwidth]{figure/graphic-HierarchicalSim2} 
\includegraphics[width=.16\textwidth]{figure/graphic-HierarchicalSim3} 
\includegraphics[width=.16\textwidth]{figure/graphic-HierarchicalSim4} 
\includegraphics[width=.16\textwidth]{figure/graphic-HierarchicalSim5} 
\includegraphics[width=.16\textwidth]{figure/graphic-HierarchicalSim6} 
\includegraphics[width=.16\textwidth]{figure/graphic-HierarchicalSim7} 
\includegraphics[width=.16\textwidth]{figure/graphic-HierarchicalSim8} 
\includegraphics[width=.16\textwidth]{figure/graphic-HierarchicalSim9} 
\includegraphics[width=.16\textwidth]{figure/graphic-HierarchicalSim10} 
\includegraphics[width=.16\textwidth]{figure/graphic-HierarchicalSim11} 
\includegraphics[width=.16\textwidth]{figure/graphic-HierarchicalSim12} 
\includegraphics[width=.16\textwidth]{figure/graphic-HierarchicalSim13} 
\includegraphics[width=.16\textwidth]{figure/graphic-HierarchicalSim14} 
\includegraphics[width=.16\textwidth]{figure/graphic-HierarchicalSim15} 
\includegraphics[width=.16\textwidth]{figure/graphic-HierarchicalSim16} 
\includegraphics[width=.16\textwidth]{figure/graphic-HierarchicalSim17} 
\includegraphics[width=.16\textwidth]{figure/graphic-HierarchicalSim18} 
\includegraphics[width=.16\textwidth]{figure/graphic-HierarchicalSim19} 
\includegraphics[width=.16\textwidth]{figure/graphic-HierarchicalSim20} 
\includegraphics[width=.16\textwidth]{figure/graphic-HierarchicalSim21} 
\includegraphics[width=.16\textwidth]{figure/graphic-HierarchicalSim22} 
\includegraphics[width=.16\textwidth]{figure/graphic-HierarchicalSim23} 
\includegraphics[width=.16\textwidth]{figure/graphic-HierarchicalSim24} 
\includegraphics[width=.16\textwidth]{figure/graphic-HierarchicalSim25} 
\includegraphics[width=.16\textwidth]{figure/graphic-HierarchicalSim26} 
\includegraphics[width=.16\textwidth]{figure/graphic-HierarchicalSim27} 
\includegraphics[width=.16\textwidth]{figure/graphic-HierarchicalSim28} 
\includegraphics[width=.16\textwidth]{figure/graphic-HierarchicalSim29} 
\includegraphics[width=.16\textwidth]{figure/graphic-HierarchicalSim30} 
\includegraphics[width=.16\textwidth]{figure/graphic-HierarchicalSim31} 
\includegraphics[width=.16\textwidth]{figure/graphic-HierarchicalSim32} 
\includegraphics[width=.16\textwidth]{figure/graphic-HierarchicalSim33} 
\includegraphics[width=.16\textwidth]{figure/graphic-HierarchicalSim34} 
\includegraphics[width=.16\textwidth]{figure/graphic-HierarchicalSim35} 
\includegraphics[width=.16\textwidth]{figure/graphic-HierarchicalSim36} 
\includegraphics[width=.16\textwidth]{figure/graphic-HierarchicalSim37} 
\includegraphics[width=.16\textwidth]{figure/graphic-HierarchicalSim38} 
\includegraphics[width=.16\textwidth]{figure/graphic-HierarchicalSim39} 
\includegraphics[width=.16\textwidth]{figure/graphic-HierarchicalSim40} 
\includegraphics[width=.16\textwidth]{figure/graphic-HierarchicalSim41} 
\includegraphics[width=.16\textwidth]{figure/graphic-HierarchicalSim42} 
\includegraphics[width=.16\textwidth]{figure/graphic-HierarchicalSim43} 
\includegraphics[width=.16\textwidth]{figure/graphic-HierarchicalSim44} 
\includegraphics[width=.16\textwidth]{figure/graphic-HierarchicalSim45} 
\includegraphics[width=.16\textwidth]{figure/graphic-HierarchicalSim46} 
\includegraphics[width=.16\textwidth]{figure/graphic-HierarchicalSim47} 
\includegraphics[width=.16\textwidth]{figure/graphic-HierarchicalSim48} 
\includegraphics[width=.16\textwidth]{figure/graphic-HierarchicalSim49} 
\includegraphics[width=.16\textwidth]{figure/graphic-HierarchicalSim50} 
\includegraphics[width=.16\textwidth]{figure/graphic-HierarchicalSim51} 
\includegraphics[width=.16\textwidth]{figure/graphic-HierarchicalSim52} 

}



\end{knitrout}


Hierarchical models are virtually identical as random coefficients models, but are more pure Bayesian in thought process and methodology.
\begin{lstlisting}
%let notedir = Z:\Dropbox\Active\STAT451\Notes\data;
filename data "&notedir./data/10randomcoef.dat";
filename outfile "&notedir./output/2013-03-07.html";
ods listing close;

data monkeyRatio;
    infile data;
    input  obs variety yield moisture;
run;

ODS GRAPHICS on / imagename="2013-03-07-Plots";
ods html body=outfile (url=none)
         GPATH="&notedir.\figures\";

proc mcmc data=wheat output=wheat_chains nmc=100000 nbi=20000 thin=10 seed=1234 monitor=(_parms_);
    diag(ess autocorr rl) propcov=quanew;
    parms mu0 30 mu1 0 s2 1;
    parms s2int 30 s2slp .1;
    random int ~ normal(mu0,var=s2int) subject=variety monitor=(int_1 int_2);
    random slp ~ normal(mu1,var=s2slp) subject=variety;
    prior mu0 ~ normal(30, var=100);
    prior mu1 ~ normal(0, var=0.1);
    prior s2 ~ dunif(0,3);
    prior s2int ~ dunif(0,50);
    prior s2slp ~ dunif(0,.2);
    mu = int + slp*moisture;
    model yield ~ normal(mu,var=s2);
run;
ods html close;
\end{lstlisting}
% section hierarchical_models (end)
\renewcommand{\dateTaken}{March 12, 2013}
\daysep

\section{Exam Debriefing} % (fold)
\label{sec:exam_debriefing}
Covergence Diagnostics should have included:
\begin{itemize}
    \item Traceplots
    \item Autocorrelation
    \item Effective sample size
    \item Raftery-lewis
\end{itemize}
Model should have been a model with six means and six variances

\begin{tabular}{cccccc}
 &  &   & Gain & &\\
 &  & 1 & 2    & 3 &\\
 Duration & 1 &$\mu_{1}$ & $\mu_{2}$& $\mu_{3}$ & $\frac{\mu_{1}+\mu_{2}+\mu_{3}}{3}$ \\
 & 2 &$\mu_{4}$ & $\mu_{5}$& $\mu_{6}$&$\frac{\mu_{4}+\mu_{5}+\mu_{6}}{3}$ \\
 & & $\frac{\mu_{1}+\mu{4}}{2}$ & $\frac{\mu_{2}+\mu{5}}{2}$ & $\frac{\mu_{3}+\mu{6}}{2}$ &
\end{tabular}

Next exam will probably have
\begin{itemize}
    \item a random effects and or hierarchal model
    \item Non-normal likelihood
\end{itemize}
% section exam_debriefing (end)
\section{Logistic Regression} % (fold)
\label{sec:logistic_regression}
What we have done so far:
\begin{itemize}
    \item ANOVA
    \item Regression
    \item ANCOVA --- Both Categorical and Continuous predictors
    \item Mixed --- More than one source of variability
\end{itemize}

\begin{tabular}{c c| c c}
 &  & \textbf{Response} & \\
 &  & Categorical & Continuous \\
 \hline
 \textbf{Predictors} & Categorical & & ANOVA \\
 & Continuous & & Regression
\end{tabular}

$y=\binomd(n,\pi)$, with $0 \le \pi \le 1$

Odds: $\frac{\pi}{1-\pi}$

Sports ``3 to 1'' is 3 failures for every one success or $\pi=.25$

$Odds > 0$, 1 should be the midpoint

odds have a skewed distribution

\begin{tabular}{c|c|c}
 \hline
 probability & Odds & $\log(\text{Odds)}$\\
 $\pi$ & Success-Failure & $\log(S/F)$\\
 \hline
 $\frac{1}{5}$ & 1 - 1 & 0\\
 $\frac{1}{5}$ & 1 - 4 & -1.39\\
 $\frac{4}{5}$ & 4 - 1 & 1.39\\
 \hline
\end{tabular}

\begin{knitrout}\scriptsize
\definecolor{shadecolor}{rgb}{0.969, 0.969, 0.969}\color{fgcolor}\begin{kframe}
\begin{alltt}
chd <- \hlfunctioncall{rbind}(\hlfunctioncall{c}(17, 274, 0, 0, 0), \hlfunctioncall{c}(15, 122, 0, 1, 0), \hlfunctioncall{c}(7, 59, 0, 0, 1), \hlfunctioncall{c}(5, 32, 
    0, 1, 1), \hlfunctioncall{c}(1, 8, 1, 9, 9), \hlfunctioncall{c}(9, 39, 1, 1, 0), \hlfunctioncall{c}(3, 17, 1, 0, 1), \hlfunctioncall{c}(14, 58, 1, 
    1, 1))
\hlfunctioncall{colnames}(chd) <- \hlfunctioncall{c}(\hlstring{"CHD"}, \hlstring{"nRisk"}, \hlstring{"Cat"}, \hlstring{"agegrp"}, \hlstring{"abECG"})
tmt <- \hlfunctioncall{seq}(8)
chd <- \hlfunctioncall{cbind}(chd, tmt)
CHD <- chd[, 1]
nRisk <- chd[, 2]
tmt <- chd[, 6]
\end{alltt}
\end{kframe}
\end{knitrout}


Cell means model
\begin{knitrout}\scriptsize
\definecolor{shadecolor}{rgb}{0.969, 0.969, 0.969}\color{fgcolor}\begin{kframe}
\begin{alltt}
mdl <- ' model \{
    \hlfunctioncall{for} (i in 1:8) \{
        CHD[i] ~ \hlfunctioncall{dbin}(p[i], nRisk[i]);
            \hlfunctioncall{logit}(p[i]) <- b[tmt[i]];
            b[i] ~ \hlfunctioncall{dnorm}(0, .1);
    \}
\}
'
\hlfunctioncall{writeLines}(mdl,\hlstring{'code/LogitBinomial.jags'})
\end{alltt}
\end{kframe}
\end{knitrout}

\begin{knitrout}\scriptsize
\definecolor{shadecolor}{rgb}{0.969, 0.969, 0.969}\color{fgcolor}\begin{kframe}
\begin{alltt}
jags.data <- \hlfunctioncall{c}(\hlstring{"CHD"}, \hlstring{"nRisk"}, \hlstring{"tmt"})
jags.params <- \hlfunctioncall{c}(\hlstring{"b"})

\hlcomment{# innits <- list(list('p' <- runif(8,0,1)))}

logit.jags <- \hlfunctioncall{jags}(data = jags.data, parameters.to.save = jags.params, model.file = \hlstring{"code/LogitBinomial.jags"}, 
    n.iter = 12000, n.burnin = 2000, n.chains = 1, n.thin = 1)
\end{alltt}


{\ttfamily\noindent\itshape\color{messagecolor}{\#\# module glm loaded}}\begin{verbatim}
## Compiling model graph
##    Resolving undeclared variables
##    Allocating nodes
##    Graph Size: 42
## 
## Initializing model
\end{verbatim}
\end{kframe}
\end{knitrout}

\renewcommand{\dateTaken}{March 14, 2013}
\daysep
\begin{align*}
    \text{logit}(p_{i})&=b_{i}\\
    \log(\frac{p_{i}}{1-p_{i}})&=b_{i}\\
    \frac{p_{i}}{1-p_{i}}&=e^{b_{i}}\\
    p_{i} &= e^{b_{i}}-pe^{b_{i}}\\
    p_{i} + p_{i}e^{b_{i}}&=e^{b_{i}}\\
    p_{i} &= \frac{e^{b_{i}}}{1+e^{b_{i}}} \frac{e^{-b_{i}}}{e^{-b_{i}}}\\
    p_{i} &= \frac{1}{e^{-b_{i}+1}}\\
\end{align*}
\begin{knitrout}\scriptsize
\definecolor{shadecolor}{rgb}{0.969, 0.969, 0.969}\color{fgcolor}\begin{kframe}
\begin{alltt}
logit.sim <- \hlfunctioncall{as.mcmc}(logit.jags)
p1 <- 1/(\hlfunctioncall{exp}(-logit.sim[, 1]) + 1)
\hlfunctioncall{plot}(\hlfunctioncall{density}(p1))
\end{alltt}
\end{kframe}

{\centering \includegraphics[width=.4\textwidth]{figure/graphic-unnamed-chunk-47} 

}



\end{knitrout}

\begin{knitrout}\scriptsize
\definecolor{shadecolor}{rgb}{0.969, 0.969, 0.969}\color{fgcolor}\begin{kframe}
\begin{alltt}
jags.params <- \hlfunctioncall{c}(\hlstring{"b"}, \hlstring{"p"})
logit.jags <- \hlfunctioncall{jags}(data = jags.data, parameters.to.save = jags.params, model.file = \hlstring{"code/LogitBinomial.jags"}, 
    n.iter = 12000, n.burnin = 2000, n.chains = 1, n.thin = 1)
\end{alltt}
\begin{verbatim}
## Compiling model graph
##    Resolving undeclared variables
##    Allocating nodes
##    Graph Size: 42
## 
## Initializing model
\end{verbatim}
\begin{alltt}
logit.sim <- \hlfunctioncall{as.mcmc}(logit.jags)
\end{alltt}
\end{kframe}
\end{knitrout}

factorial design 

Cat-N
\begin{tabular}{cccc}
      &  & abEUG & \\
      &  & Y & O\\
abECG & N & $\mu_1$ & $\mu_2$ \\
      & Y & $\mu_3$ & $\mu_4$ \\
\end{tabular}

Cat-Y
\begin{tabular}{cccc}
      &  & abEUG & \\
      &  & Y & O\\
abECG & N & $\mu_5$ & $\mu_6$ \\
      & Y & $\mu_7$ & $\mu_8$ \\
\end{tabular}

To find the marginals for cat
\begin{align}
    \text{Cat-n} = \frac{\mu_1+\mu_2+\mu_3+\mu_4}{4}
    \text{Cat-y} = \frac{\mu_1+\mu_2+\mu_3+\mu_4}{4}
\end{align}
% Dr Reese came to class today
% \begin{itemize}
%     \item B is a log of an odds
%     \item B is a ratio
% \end{itemize}
\begin{knitrout}\scriptsize
\definecolor{shadecolor}{rgb}{0.969, 0.969, 0.969}\color{fgcolor}\begin{kframe}
\begin{alltt}
p1 <- logit.sim[, 10]
p2 <- logit.sim[, 11]
p3 <- logit.sim[, 12]
p4 <- logit.sim[, 13]
p5 <- logit.sim[, 14]
p6 <- logit.sim[, 15]
p7 <- logit.sim[, 16]
p8 <- logit.sim[, 17]
\end{alltt}
\end{kframe}
\end{knitrout}

Find the marginals
\begin{knitrout}\scriptsize
\definecolor{shadecolor}{rgb}{0.969, 0.969, 0.969}\color{fgcolor}\begin{kframe}
\begin{alltt}
catno <- (p1 + p2 + p3 + p4)/4
catyes <- (p5 + p6 + p7 + p8)/4
\hlfunctioncall{plot}(\hlfunctioncall{density}(catno), xlim = \hlfunctioncall{c}(0.05, 0.4), lwd = 2)
\hlfunctioncall{lines}(\hlfunctioncall{density}(catyes), lwd = 2, col = \hlstring{"red"})

young <- (p1 + p3 + p5 + p7)/4
old <- (p2 + p4 + p6 + p8)/4
\hlfunctioncall{plot}(\hlfunctioncall{density}(young), xlim = \hlfunctioncall{c}(0.05, 0.4), ylim = \hlfunctioncall{c}(0, 15), lwd = 2)
\hlfunctioncall{lines}(\hlfunctioncall{density}(old), lwd = 2, col = \hlstring{"red"})

abNo <- (p1 + p2 + p5 + p6)/4
abYes <- (p3 + p4 + p7 + p8)/4
\end{alltt}
\end{kframe}

{\centering \includegraphics[width=.4\textwidth]{figure/graphic-unnamed-chunk-501} 
\includegraphics[width=.4\textwidth]{figure/graphic-unnamed-chunk-502} 

}



\end{knitrout}


To compare old to young we use an odds ratio or a risk ratio
\begin{knitrout}\scriptsize
\definecolor{shadecolor}{rgb}{0.969, 0.969, 0.969}\color{fgcolor}\begin{kframe}
\begin{alltt}
ORage <- (old/(1 - old))/(young/(1 - young))
\hlfunctioncall{mean}(ORage > 1)
\end{alltt}
\begin{verbatim}
## [1] 0.9027
\end{verbatim}
\begin{alltt}

RRage <- old/young
\hlfunctioncall{mean}(RRage > 1)
\end{alltt}
\begin{verbatim}
## [1] 0.9027
\end{verbatim}
\begin{alltt}

ORab <- (abYes/(1 - abYes))/(abNo/(1 - abNo))
ORCat <- (catyes/(1 - catyes))/(catno/(1 - catno))

\hlfunctioncall{mean}(ORab)
\end{alltt}
\begin{verbatim}
## [1] 1.402
\end{verbatim}
\begin{alltt}
\hlfunctioncall{mean}(ORCat)
\end{alltt}
\begin{verbatim}
## [1] 2.001
\end{verbatim}
\begin{alltt}
\hlfunctioncall{par}(mfrow = \hlfunctioncall{c}(2, 2))
xlims <- \hlfunctioncall{c}(0, 0.5)
ylims <- \hlfunctioncall{c}(0, 20)
\hlfunctioncall{plot}(\hlfunctioncall{density}(catno), xlim = xlims, ylim = ylims)
\hlfunctioncall{lines}(\hlfunctioncall{density}(catyes), col = \hlstring{"blue"})
\hlfunctioncall{plot}(\hlfunctioncall{density}(abNo), xlim = xlims, ylim = ylims)
\hlfunctioncall{lines}(\hlfunctioncall{density}(abYes), col = \hlstring{"blue"})
\hlfunctioncall{plot}(\hlfunctioncall{density}(young), xlim = xlims, ylim = ylims)
\hlfunctioncall{lines}(\hlfunctioncall{density}(old), col = \hlstring{"blue"})
\hlfunctioncall{par}(mfrow = \hlfunctioncall{c}(1, 1))
\end{alltt}
\end{kframe}

{\centering \includegraphics[width=\textwidth]{figure/graphic-unnamed-chunk-51} 

}



\end{knitrout}


two way interaction for abnormal ECG and age averaged over catacolamines
\begin{align*}
    \mu_{1} + \mu_{5} - \mu_{2} - \mu_{6} - \mu_{3} - \mu_{7} + \mu_{4} + \mu_{8} = 0
\end{align*}

to average across abnormal ECG, rewrite the tables
abECG-N
\begin{tabular}{cccc}
      &  & abEUG & \\
      &  & Y & O\\
Cat   & N & $\mu_1$ & $\mu_2$ \\
      & Y & $\mu_5$ & $\mu_6$ \\
\end{tabular}

abECG-Y
\begin{tabular}{cccc}
      &  & abEUG & \\
      &  & Y & O\\
Cat   & N & $\mu_3$ & $\mu_4$ \\
      & Y & $\mu_7$ & $\mu_8$ \\
\end{tabular}
\begin{align*}
    \mu_{1} + \mu_{3} - \mu_{2} - \mu_{4} - \mu_{5} - \mu_{7} + \mu_{6} + \mu_{8} = 0
\end{align*}
\begin{knitrout}\scriptsize
\definecolor{shadecolor}{rgb}{0.969, 0.969, 0.969}\color{fgcolor}\begin{kframe}
\begin{alltt}
intageab <- p1 + p5 - p2 - p6 - p3 - p7 + p4 + p8
\hlfunctioncall{plot}(\hlfunctioncall{density}(intageab))
\end{alltt}
\end{kframe}

{\centering \includegraphics[width=.4\textwidth]{figure/graphic-unnamed-chunk-52} 

}



\end{knitrout}

\renewcommand{\dateTaken}{March 19, 2013}
\daysep
\subsection{Logistic Regression In SAS} % (fold)
\label{sub:logistic_regression_in_sas}

JAGS freefloating, SAS contstrained

\begin{tabular}{cccc}
\hline
Cat-N &  & Age & \\
      &  & Y & O\\
ECG   & N & $p_1$, .07, .06 & $p_2$, .12, .12 \\
      & Y & $p_3$, .10, .12 & $p_4$, .16, .16 \\
      \hline
\end{tabular}

\begin{tabular}{cccc}
\hline
Cat-Y &  & Age & \\
      &  & Y & O\\
ECG   & N & $p_5$, .12, .15 & $p_6$, .20, .23 \\
      & Y & $p_7$, .17, .18& $p_8$,  .26, .24\\
      \hline
\end{tabular}

SAS:
\begin{align*}
\log\left(\frac{p}{1-p}\right) & =\beta_{0}+\beta_{\text{cat}}+\beta_{\text{age}}+\beta_{\text{ecg}}\\
    p & =\frac{1}{1+e^{-x\beta}}\\
    p_{5} &= \frac{1}{1+e^{-\beta_{0}-\beta_{\text{cat}}}}
\end{align*}
This model has 4 degrees of freedom

\begin{align*}
    X = \begin{bmatrix}
        1 & 0 & 0 & 0\\
        1 & 0 & 1 & 0\\
        1 & 0 & 0 & 1\\
        1 & 0 & 1 & 1\\
        1 & 1 & 0 & 0\\
        1 & 1 & 1 & 0\\
        1 & 1 & 0 & 1\\
        1 & 1 & 1 & 1\\
    \end{bmatrix}
\end{align*}
\begin{knitrout}\scriptsize
\definecolor{shadecolor}{rgb}{0.969, 0.969, 0.969}\color{fgcolor}\begin{kframe}
\begin{alltt}
1/(1 + \hlfunctioncall{exp}(2.6263 - 0.6199))
\end{alltt}
\begin{verbatim}
## [1] 0.1185
\end{verbatim}
\end{kframe}
\end{knitrout}


JAGS Model
\begin{align*}
    \log\left(\frac{p_{i}}{1-p_{i}}\right)=b_{i}\qquad i=\{1,\ldots,8\}
\end{align*}
this model has 8 degrees of freedom

\begin{knitrout}\scriptsize
\definecolor{shadecolor}{rgb}{0.969, 0.969, 0.969}\color{fgcolor}\begin{kframe}
\begin{alltt}
mdl <- ' model \{
    \hlfunctioncall{for} (i in 1:8) \{
        CHD[i] ~ \hlfunctioncall{dbin}(p[i], nRisk[i]);
            \hlfunctioncall{logit}(p[i]) <- bint + bcat*cat[i] + bage*age[i] + becg*ecg[i];
    \}
    
    bint ~ \hlfunctioncall{dnorm}(0, .1);
    bcat ~ \hlfunctioncall{dnorm}(0, .1);
    bage ~ \hlfunctioncall{dnorm}(0, .1);
    becg ~ \hlfunctioncall{dnorm}(0, .1);
\}
'
\hlfunctioncall{writeLines}(mdl,\hlstring{'code/LogitReducedModel.jags'})
\end{alltt}
\end{kframe}
\end{knitrout}

\begin{knitrout}\scriptsize
\definecolor{shadecolor}{rgb}{0.969, 0.969, 0.969}\color{fgcolor}\begin{kframe}
\begin{alltt}
CHD <- chd[, 1]
nRisk <- chd[, 2]
tmt <- chd[, 6]
cat <- chd[, 3]
age <- chd[, 4]
ecg <- chd[, 5]

jags.data <- \hlfunctioncall{c}(\hlstring{"CHD"}, \hlstring{"nRisk"}, \hlstring{"cat"}, \hlstring{"age"}, \hlstring{"ecg"})
jags.params <- \hlfunctioncall{c}(\hlstring{"bint"}, \hlstring{"bcat"}, \hlstring{"bage"}, \hlstring{"becg"}, \hlstring{"p"})

\hlfunctioncall{set.seed}(12)
logit.jags <- \hlfunctioncall{jags}(data = jags.data, parameters.to.save = jags.params, model.file = \hlstring{"code/LogitReducedModel.jags"}, 
    n.iter = 12000, n.burnin = 2000, n.chains = 1, n.thin = 1)
\end{alltt}
\begin{verbatim}
## Compiling model graph
##    Resolving undeclared variables
##    Allocating nodes
##    Graph Size: 70
## 
## Initializing model
\end{verbatim}
\end{kframe}
\end{knitrout}

SAS's Deviance is calculate as $\text{Dbar}-\text{Dmean}$

computing $pD=\frac{\var(\text{deviance})}{2}$ is a better way to calculate it in general
% subsection logistic_regression_in_sas (end)
\subsection{Random Effects} % (fold)
\label{sub:random_effects}
\begin{knitrout}\scriptsize
\definecolor{shadecolor}{rgb}{0.969, 0.969, 0.969}\color{fgcolor}\begin{kframe}
\begin{alltt}
seeds <- \hlfunctioncall{read.table}(\hlstring{"data/seednew.dat"}, stringsAsFactors = FALSE, skip = 1, col.names = \hlfunctioncall{c}(\hlstring{"r"}, 
    \hlstring{"n"}, \hlstring{"tmt"}, \hlstring{"nseed"}, \hlstring{"ntype"}))
\hlfunctioncall{sum}(seeds$n[seeds$tmt == 1])
\end{alltt}
\begin{verbatim}
## [1] 272
\end{verbatim}
\begin{alltt}
\hlfunctioncall{sum}(seeds$r[seeds$tmt == 1])
\end{alltt}
\begin{verbatim}
## [1] 99
\end{verbatim}
\begin{alltt}

\hlfunctioncall{var}(seeds$r[seeds$tmt == 1])
\end{alltt}
\begin{verbatim}
## [1] 40.7
\end{verbatim}
\end{kframe}
\end{knitrout}

Is the variability reasonable?

\begin{knitrout}\scriptsize
\definecolor{shadecolor}{rgb}{0.969, 0.969, 0.969}\color{fgcolor}\begin{kframe}
\begin{alltt}
mdl <- 'model \{
    \hlfunctioncall{for} (i in 1:21) \{
        r[i] ~ \hlfunctioncall{dbin}(p[i], n[i])
        \hlfunctioncall{logit}(p[i]) <- b[tmt[i]] + e[i]
    \}

    \hlfunctioncall{for} (i in 1:4) \{
        b[i] ~ \hlfunctioncall{dnorm}(0, 1)
    \}

    \hlfunctioncall{for} (i in 1:21) \{
        e[i] ~ \hlfunctioncall{dnorm}(0, prec)
    \}
    s2 ~ \hlfunctioncall{dunif}(0,2);
    prec <- 1/s2
\}
'
\hlfunctioncall{writeLines}(mdl, \hlstring{'code/LogitMixedModel.jags'})
mdl <- 'model \{
    \hlfunctioncall{for} (i in 1:21) \{
        r[i] ~ \hlfunctioncall{dbin}(p[i], n[i])
        \hlfunctioncall{logit}(p[i]) <- b[tmt[i]]
    \}

    \hlfunctioncall{for} (i in 1:4) \{
        b[i] ~ \hlfunctioncall{dnorm}(0, 1)
    \}
\}
'
\hlfunctioncall{writeLines}(mdl, \hlstring{'code/Logit09Simplified.jags'})
\end{alltt}
\end{kframe}
\end{knitrout}

\begin{knitrout}\scriptsize
\definecolor{shadecolor}{rgb}{0.969, 0.969, 0.969}\color{fgcolor}\begin{kframe}
\begin{alltt}
r <- seeds[, 1]
n <- seeds[, 2]
tmt <- seeds[, 3]
jags.data <- \hlfunctioncall{c}(\hlstring{"r"}, \hlstring{"n"}, \hlstring{"tmt"})
jags.params <- \hlfunctioncall{c}(\hlstring{"b"}, \hlstring{"e"})

seed.jags <- \hlfunctioncall{jags}(data = jags.data, parameters.to.save = jags.params, model.file = \hlstring{"code/LogitMixedModel.jags"}, 
    n.iter = 12000, n.burnin = 2000, n.chains = 1, n.thin = 1)
\end{alltt}


{\ttfamily\noindent\itshape\color{messagecolor}{\#\# module glm loaded}}\begin{verbatim}
## Compiling model graph
##    Resolving undeclared variables
##    Allocating nodes
##    Graph Size: 135
## 
## Initializing model
\end{verbatim}
\begin{alltt}

jags.params <- \hlfunctioncall{c}(\hlstring{"b"})
seedsimple.jags <- \hlfunctioncall{jags}(data = jags.data, parameters.to.save = jags.params, model.file = \hlstring{"code/Logit09Simplified.jags"}, 
    n.iter = 12000, n.burnin = 2000, n.chains = 1, n.thin = 1)
\end{alltt}
\begin{verbatim}
## Compiling model graph
##    Resolving undeclared variables
##    Allocating nodes
##    Graph Size: 73
## 
## Initializing model
\end{verbatim}
\begin{alltt}
seed.jags
\end{alltt}
\begin{verbatim}
## Inference for Bugs model at "code/LogitMixedModel.jags", fit using jags,
##  1 chains, each with 12000 iterations (first 2000 discarded)
##  n.sims = 10000 iterations saved
##          mu.vect sd.vect   2.5%    25%    50%     75%   97.5%
## b[1]      -0.522   0.221 -0.948 -0.665 -0.525  -0.385  -0.068
## b[2]       0.816   0.221  0.384  0.676  0.808   0.953   1.270
## b[3]      -0.457   0.268 -1.008 -0.625 -0.446  -0.281   0.041
## b[4]       0.017   0.254 -0.507 -0.143  0.028   0.186   0.503
## e[1]      -0.287   0.298 -0.937 -0.466 -0.264  -0.082   0.232
## e[2]      -0.016   0.265 -0.559 -0.177 -0.010   0.149   0.506
## e[3]      -0.271   0.268 -0.846 -0.436 -0.252  -0.086   0.207
## e[4]       0.338   0.289 -0.174  0.136  0.318   0.518   0.946
## e[5]       0.144   0.282 -0.401 -0.035  0.134   0.314   0.728
## e[6]       0.113   0.381 -0.585 -0.128  0.086   0.329   0.948
## e[7]       0.068   0.261 -0.442 -0.099  0.061   0.230   0.605
## e[8]       0.233   0.276 -0.264  0.045  0.213   0.403   0.820
## e[9]      -0.171   0.275 -0.748 -0.347 -0.157   0.012   0.335
## e[10]     -0.217   0.264 -0.767 -0.383 -0.204  -0.040   0.278
## e[11]      0.128   0.350 -0.517 -0.093  0.103   0.326   0.919
## e[12]      0.170   0.338 -0.451 -0.050  0.144   0.373   0.913
## e[13]     -0.120   0.310 -0.766 -0.312 -0.108   0.073   0.494
## e[14]     -0.210   0.319 -0.904 -0.406 -0.187  -0.002   0.375
## e[15]      0.292   0.311 -0.258  0.079  0.265   0.480   0.978
## e[16]     -0.221   0.413 -1.168 -0.433 -0.172   0.042   0.459
## e[17]     -0.314   0.368 -1.148 -0.526 -0.275  -0.068   0.315
## e[18]      0.088   0.292 -0.465 -0.098  0.076   0.266   0.706
## e[19]     -0.010   0.296 -0.595 -0.192 -0.016   0.171   0.601
## e[20]      0.302   0.301 -0.237  0.097  0.282   0.484   0.956
## e[21]     -0.073   0.363 -0.828 -0.284 -0.062   0.148   0.641
## deviance  98.860   5.637 88.953 94.822 98.517 102.524 110.858
## 
## DIC info (using the rule, pD = var(deviance)/2)
## pD = 15.9 and DIC = 114.7
## DIC is an estimate of expected predictive error (lower deviance is better).
\end{verbatim}
\begin{alltt}
seedsimple.jags
\end{alltt}
\begin{verbatim}
## Inference for Bugs model at "code/Logit09Simplified.jags", fit using jags,
##  1 chains, each with 12000 iterations (first 2000 discarded)
##  n.sims = 10000 iterations saved
##          mu.vect sd.vect    2.5%     25%     50%     75%   97.5%
## b[1]      -0.552   0.126  -0.800  -0.637  -0.551  -0.467  -0.308
## b[2]       0.796   0.125   0.556   0.712   0.795   0.879   1.043
## b[3]      -0.406   0.183  -0.766  -0.529  -0.404  -0.283  -0.056
## b[4]       0.125   0.168  -0.207   0.010   0.126   0.238   0.454
## deviance 111.575   2.873 108.077 109.515 110.875 112.916 118.965
## 
## DIC info (using the rule, pD = var(deviance)/2)
## pD = 4.1 and DIC = 115.7
## DIC is an estimate of expected predictive error (lower deviance is better).
\end{verbatim}
\end{kframe}
\end{knitrout}

% subsection random_effects (end)
\renewcommand{\dateTaken}{March 21, 2013}
\daysep
\lstinsertlisting[language=SAS]{2013-03-21_SAS.sas}
% section logistic_regression (end)
\section{Poisson Likelihood} % (fold)
\label{sec:poisson_likelihood}
11epi.dat Epilepsy 
\begin{itemize}
    \item ``pid'' --- Patient ID
    \item ``scount'' --- Seizure Count 
    \item ``tmt'' --- Treatment
    \item ``base'' --- Base Effect
    \item ``age'' --- Age
    \item ``visit'' 
\end{itemize}

$S$ is the number of seizures (scount)
\begin{align*}
    S_{i} &\sim \poisd(\lambda)\\
    \lambda = a_{0} + a_{\text{age}}*\text{age}+ a_{\text{tmt}}*\text{tmt}+ a_{\text{base}}*\text{base}
\end{align*}
% section poisson_likelihood (end)
\end{document}
